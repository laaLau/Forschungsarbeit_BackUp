\documentclass{scrartcl}
% Pakete laden 
\usepackage{graphicx}
\usepackage{placeins}
\usepackage[utf8]{inputenc}
\usepackage[T1]{fontenc}
\usepackage[ngerman]{babel}
\usepackage[autostyle=true,german=quotes]{csquotes}
\usepackage[default,osfigures,scale=1]{opensans} 
% Verlinke Zitate
\usepackage[hidelinks]{hyperref}
\usepackage[doublespacing]{setspace}
% Verlinke das Inhaltsverzeichnis
\hypersetup{linktocpage}
\usepackage{csquotes}

%\newcommand*{\Bildquelle}{% 
 % \newline \footnotesize Quelle: }

\usepackage[notes,backend=biber]{biblatex-chicago}

%\DeclareLanguageMapping{ngerman}{ngerman-apa}
\addbibresource{mendeley_v3.bib}
% Keine Einrückung
\setlength\parindent{0pt}

\title{Ist urbane Landwirtschaft eine zukünftige Möglichkeit der (ökologisch) nachhaltigen Lebensmittelversorgung in Städten?}
\author{Laura Schillke, Sebastian Meidel, Mekong Lam }
\date{9. November 2018}

% gepunktetes Inhaltsverzeichnis
\usepackage{tocstyle}
\newtocstyle[KOMAlike][leaders]{alldotted}{}
\usetocstyle{alldotted}
\usepackage{acronym}

%Dokument beginnt
\begin{document}
\maketitle


\newpage

\setcounter{tocdepth}{2}
\tableofcontents 



\newpage
\section{Abkürzungsverzeichnis}

\begin{acronym}
	\acro{ul}[UL]{Urbane Landwirtschaft}
	\acro{ug}[UG]{Urban Gardening}
	\acro{fao}[FAO]{Ernährungs- und Landwirtschaftsorganisation der Vereinten Nationen} 
	\acro{vl}[VL]{Vertikale Landwirtschaft}
\end{acronym}


\newpage

\section{Abstract}
Urbane Landwirtschaft ist eine Thematik, die seit längerem im deutschsprachigen Raum mediale Aufmerksamkeit erhält. In dieser Forschungsarbeit wird dieser Begriff im Kontext einer Fragestellung untersucht: Ist urbane Landwirtschaft nur ein Ausdruck der Landlust der Stadtbewohner oder wird er den zukünftig wachsenden Lebensmittelbedarf der Stadtbevölkerung lösen? Die besondere Anforderung ist in diesem Zusammenhang der Aspekt der Nachhaltigkeit. Zu Beginn dieser Forschungsarbeit werden die Grundlagen des Themas erläutert. Darauf aufbauend wird die Entwicklung der Lebensmittelproduktion untersucht und wie sie den Bezug zwischen Lebensmitteln, Erzeugern und dem Konsumenten im städtischen Kontext verändert hat. Des Weiteren wird die Bedeutung urbaner Landwirtschaft (emotional und pragmatisch) von der Vergangenheit bis heute dargestellt. Im Anschluss soll ein modernes Beispiel stadtnaher Lebensmittelproduktion zeigen, wie nachhaltig dieser Versorgungsansatz sein kann und die Fragestellung beantworten.

\section{Forschungsansatz}

\section{Motivation}

\begin{figure}[htbp]
\centering
\hspace*{-1,8cm}   
\includegraphics[width=18cm]{image_folder/The_earth_at_night.jpg}
\caption{Verstädterung der Welt}
\label{fig:sattelitenbild}
\end{figure}

Abbildung \ref{fig:sattelitenbild} zeigt anhand der beleuchteten Flächen, wo sich derzeit viele Städte befinden. Diese beleuchtete Fläche wird sich vergrößern. Prognosen zufolge wird die  Weltbevölkerung im Jahr 2050 von derzeit sieben Milliarden Menschen um weitere zwei Milliarden steigen. Allerdings werden davon 68 \% in Städten leben.\footcites[Vgl.][o.P.]{UnitedNationsDESA2018WorldRevision}[][S.3]{UnitedNationsDepartmentofEconomicandSocialAffairs2017World2017} Die Stadtbevölkerung wächst und somit auch die bebaute Siedlungsfläche. Das stellt sich allerdings als Problem heraus, denn die Ausweitung von Siedlungsflächen steht in Konkurrenz zur nötigen Ackerfläche, die dem steigenden Lebensmittelbedarf gerecht werden muss. Wie können also Städte weiterhin versorgt werden trotz der begrenzten Anbaufläche? Desweiteren büßt die Bodenfläche vielerorts ihre Qualität aufgrund der beträchtlichen Landwirtschaftsnutzung ein, was einen weiteren Anlass liefert, über nachhaltigere Anbauformen nachzudenken.\\
\\
Passend dazu ist in Großstädten derzeit ein Begriff in aller Munde: Urban Gardening. Grüne DIY-Projekte, Tomatenanbau auf Verkehrsinseln oder große Gemeinschaftsgärten in der Stadt - eine Vielzahl von Gartenprojekten finden in Großstädten anklang, sowohl im privaten Heim als auch in Gemeinschaftsform. Der Eindruck wird erweckt, dass eine Sehnsucht zum Ländlichen - „Landlust“ - wächst. Aus pragmatischer Sicht könne eine stadtnahe Versorgung allerdings den Transportweg reduzieren und der knapper werdenden fossilen Energie entgegen kommen. Könnte dieser modische Begriff eine Lösung für die zukünftigen Herausforderungen der Lebensmittelversorgung sein? Oder handelt es sich hier nur um ein Sympton an Ermangelung von Grünflächen in Städten? 

\section{Grundlagen}

Für ein erleichtertes Verständnis der Forschungsarbeit werden zu Beginn einige Grundlagen erläutert.

\subsection{Stadt}

Grundsätzlich wird eine Stadt als eine größere verdichtete Siedlung bezeichnet, die bestimmte Funktionen und charakteristische Merkmale aufweist.\footcite{HaasDefinitionWirtschaftslexikon} Die Definition des Brockhaus schreibt einer Stadt Merkmale zu wie zum Beispiel eine eigene Versorgungs- und Verwaltungsstruktur, eine innere Gliederung oder eine höhere Bebauungs- und Verkehrsdichte. Hinzu kommen spezielle Funktionen wie politische Aufgaben (Haupstädte, Festungsstädte) oder wirtschaftliche Funktionen (Hansestädte, Hafenstädte, Karawanenstädte).\footcite{BrockhausStadt} Die statistischen Kriterien, die eine „Stadt“ vom „Land“ unterscheiden variieren von Land zu Land. Beispielsweise werden Städte in der Bundesrepublik Deutschland mit 
\begin{itemize}
\item 5.000 bis 20.000 Einwohnern als „Kleinstadt“,
\item 20.000 bis 50.000 Einwohnern als „Mittelstadt“,
\item ab 100.000 Einwohnern als „Großstadt“ 
\end{itemize}
bezeichnet \footcite{Bssr}. Dem gegenüber orientiert man sich in China an die Bevölkerungsdichte: 

\begin{displayquote}
„In the case of cities with district establishment, the city proper refers to the whole administrative area of the district if its population density is 1 500 people per kilometre or higher [...].“ \footcite[S.~2]{UnitedNations2005Table2005} 
\end{displayquote}

Unabhängig von regional unterschiedlichen Kriterien lässt sich mit steigender Einwohnerzahl und -dichte einer Stadt folgern, dass auch die Anforderungen an gewährleisteter Lebensmittelversorgung für die städtische Bevölkerung steigen. D.h. die Lebensmittelversorgung einer Großstadt zu gewährleisten ist schwieriger als die einer Kleinstadt. Noch größer ist die Herausforderung in Megastädten. Den Vereinten Nationen zufolge wird eine Megastadt (englisch: „Megacity“) als eine Stadt mit mindestens zehn Millionen Einwohnern bezeichnet. Im Jahr 2016 existierten 31 Megastädte und Prognosen zufolge werde die Anzahl im Jahr 2030 auf 41 Megastädte steigen\footcite{UnitedNations2016The2016}. Tokio zählt mit 38 Millionen Einwohnern als die größte Stadt weltweit und derzeit befinden sich die meisten Megastädte in Industrie- und Entwicklungsländern (wie in Abbildung \ref{figUrban} erkennbar). 

\begin{figure}[htbp]
\centering
\hspace*{-3cm}   
\includegraphics[width=20cm]{image_folder/CityPop_Urban.png}
\caption{Urbanisierung der Welt}
\label{figUrban}
\end{figure}

\FloatBarrier
In der Abbildung \ref{figUrban} erkennt man, dass Nord- und Südamerika sowie Europa deutlich verstädtert sind, wohingegen viele Regionen in Afrika und Asien mehr ländliche Gebiete beinhalten. In dieser Forschungsarbeit wird der Stadtbegriff im Bezug zu ihrer fähigen Lebensmittelversorgung und Nachhaltigkeit betrachtet. Daher eignet es sich Städte im Unterschied zu ländlichen Gemeinden zu beschreiben: 

\begin{displayquote} 
„The hallmarks of cities are: (a) a large population that (b) aggregates in a central location with (c) buildings and monuments that (d) represent institutions that organize and facilitate productivity.“.\footcite[S.16]{Elmqvist2013} 
\end{displayquote}  

Zusammenfassend heißt es, dass Städte sich kennzeichnen lassen durch eine große Einwohnerzahl angehäuft an einem zentralen Ort mit Bauten und Monumenten, die Institutionen repräsentieren und produktive Leistungen organisieren und ermöglichen.

\subsection{Urbane Landwirtschaft}

\subsubsection{Urban}
Um den Begriff Urban oder Urbanität innerhalb der folgenden Forschungsarbeit zu definieren, wird zum Teil die allgemeine Definition „städtisch“ als auch „für eine Stadt charakteristisch“ verwendet.\footcite[Vgl.]{DUDEN}Darüber hinaus werden soziokulturelle Aspekte wie die Verwendung des Begriffes im Zusammenhang mit politischen Sichtweisen wie z.B. Weltoffenheit nicht tiefer in die Definition einfließen. Innerhalb der Forschungsarbeit wird der Begriff des Urbanen oder die Urbanität primär mit der physischen Stadt oder „innerhalb einer Stadt“ verwendet. Randgebiete einer Stadt werden hier nicht als urban beschrieben sondern als „Peri-Urban“ (siehe Abbildung \ref{fig:urbaneEingrenzung}).\footcite[S. 140]{MullerUrbanStadt}
\begin{flushright}
[MS]
\end{flushright}
\begin{figure}[htbp]
\centering
\includegraphics[width=12cm]{image_folder/SchaubildUrbaneEingrenzungen.png}
\caption{Skizze zur räumlichen Einordnung}
\label{fig:urbaneEingrenzung}
\end{figure}
\footcite[]{Eigene Zeichnung, entstanden aus dem Vorbild von Carlos Tobisch, Oasen im Beton S.26, Abb.7}

\subsubsection{Landwirtschaft}\textit
Nach Lohrberg wird als Landwirtschaft der planmäßige Anbau von Nutzpflanzen oder die Zucht und Haltung von Tieren zur Lebensmittel- oder Rohstoffproduktion beschrieben. Die Landwirtschaft umfasse dabei den Nutzpflanzenanbau, die Tierhaltung, den Ackerbau, die Aquakultur und die Imkerei sowie die Fischerei. \footcite[S. 5]{Lohrberg2001StadtnaheFreiraumplanung} Laut Definition des BauGB § 201. wird unter Landwirtschaft „insbesondere der Ackerbau, die Wiesen- und Weidewirtschaft, einschließlich Pensionstierhaltung auf überwiegend eigener Futtergrundlage, die gartenbauliche Erzeugung, der Erwerbsobstbau, der Weinbau, die berufsmäßige Imkerei und die berufsmäßige Binnenfischerei“ verstanden.\footcite[Vgl.][§ 201]{Deutschland2017Baugesetzbuch}

\subsubsection[UL]{Urbane Landwirtschaft}
Urbane Landwirtschaft (im folgenden \acs{ul} genannt) umfasst (professionelle) landwirtschaftliche oder überwiegend gartenbauliche Aktivitäten in und am Rande von städtischen Siedlungsräumen. Dabei kann die urbane Landwirtschaft durchaus auch weltmarktorientiert sein und lässt sich nach Berges et. al. in verschiedene Typen einordnen. Die Akteure zum Beispiel bilden die Haupthandelnden einer gärtnerischen Aktivität ab. Dies können Privathaushalte, Kleingruppen, Gemeinschaften oder auch privatwirtschaftlich handelnde Unternehmen sein. \footcite[Vgl.][S. 10 f]{Berges2014UrbaneStadt}

Nach Berges lassen sich diese dann in verschiedene Ausrichtungen einordnen:

\begin{itemize}
\item Die Subsistenszausrichtung also die Selbstversorgung hat das Ziel den Nahrungsbedarf durch UL teilweise oder ganz zu decken oder auch um Zugang zu Bio-Lebensmitteln zu erhalten. Diese Ausrichtung wird oft von Privatpersonen und Privathaushalten betrieben. 
\item Bei der soziokulturellen Ausrichtung, die von Gemeinschaften und Vereinen betrieben wird, liegt der Fokus auf dem sozialen Austausch und dem Gemeinschaftswesen. 
\item Bei der kommerziellen Ausrichtung wird gezielt Einkommen generiert. Der Wachstum des Unternehmens und des Profits steht im Vordergrund. 
\end{itemize}

Parallel dazu kann festgestellt werden für wen, also auf welchen Ebenen, die Nahrungsmittel angebaut und vertrieben werden:

\begin{itemize}
\item Auf der Mikroebene steht die Verteilung der Erzeugnisse an die Familie im Vordergrund.
\item Auf der Mesoebene werden die Erzeugnisse mit einem geschlossenen Kreis an Menschen geteilt wie zum Beispiel Vereinsmitglieder oder Bekannte.
\item Auf der Makroebene werden die Erzeugnisse nicht mehr innerhalb von bestimmten Gruppen geteilt, sondern gelangen an Konsumenten, die in der Regel keinem bestimmten Kreis angehören. 
\end{itemize}

Im Fokus dieser Arbeit steht \acs{ul} auf der Meso- bis Makroebene. Auf diesen Ebenen lassen sich Formen der UL auf deren Ökonomie und Nachhaltigkeit untersuchen.

\begin{figure}[htbp]
\centering
\includegraphics[width=12cm]{image_folder/ul_typologie.png}
\caption{Darstellung der Typologie urbaner Landwirtschaft nach Berges et. al}
\label{fig:ul_typologie}
\end{figure}

\subsubsection{Urban Gardening} \label{UG}
\textit{MS}

Angefangen als Trend in wenigen Metropolen hält Urban Gardening (\acs{ug}) auch Einzug in immer mehr kleine und mittelgroße Städte. Auch die Kommunalpolitik greift mittlerweile auf das Konzept zurück. Nach Prof. Dr. Heeg ist \acs{ug} zusammengefasst ist die gärtnerische Nutzung städtischer Flächen. Dabei würde \acs{ug} nicht nur von engagierten Städtern praktiziert, sondern auch auf professioneller Ebene. Das urbane Gärtnern wird dabei häufig benutzt, um ungenutzte Flächen zu begrünen, Klimapolitik zu betreiben, städtische Räume qualitativ aufzuwerten oder Hobbygärtnern eine Fläche zu bieten. UG ist ein Konzept, welches mit aktuellen gesellschaftlichen, politischen und städtischen Entwicklungen eng zusammenhängt. \footcite[Vgl.][Geleitwort von Prof. Dr. Susanne Heeg f]{Biedermann2017UrbanStadtentwicklung}
UG stellt hierbei einen Teil der Urbanen Landwirtschaft dar. Und wird häufig auf der Mikro- bis Mesoebene praktiziert. Einen Unterschied zur Makroebene weist UG nach Tobisch insofern auf, dass es sich hierbei um eine oft stark politisch und sozial ambitionierte Bewegung handele oder die Subsistenz im Vordergrund stünde. Die UG-Bewegung lege ihren Fokus verstärkt auf soziale und ideelle Aspekte. \footcite[Vgl.][S.25-27]{CarlosTobisch2012OasenBrachflachen} Die Makroebene der Urbanen Landwirtschaft zeichnet sich im Gegenzug dazu durch einen professionellen Umgang mit dem Anbau von Lebensmitteln aus. Hier steht die Produktion für den Verbrauchermarkt im Vordergrund.

\begin{figure}[htbp]
\centering
\includegraphics[width=12cm]{image_folder/SchaubildULvsUG.png}
\caption{Eigene Zeichnung, entstanden aus dem Vorbild von Carlos Tobisch, Oasen im Beton (S.27, Abb. 1)}
\label{fig:ul_typologie}
\end{figure}

\subsection{Lebensmittelbedarf}

Essbares und Trinkbares, das Lebewesen zur Ernährung und zum Aufbau und zur Erhaltung des Organismus brauchen und zu sich nehmen wird generell unter den Begriffen Lebensmittel oder Nahrung zusammengefasst.\footcite{DudenLebensmittel}\\
\\
„Nahrung ist eine grundlegende Basis für die Existenz menschlichen Lebens“. Mit diesen Worten fasst Stierand Informationen aus von Blanckenburgs Text in „Zukunft, Welternährung - Gegenwartsprobleme und Strategien“ zusammen. \footcite[S.122f]{Stierand2008StadtLebensmittel} 
\begin{displayquote}
„Jeder Mensch benötigt eine angemessene Ernährung, um körperlich und geistig leistungsfähig zu bleiben. Auch für die Aufrechterhaltung der Gesundheit und die Abwehr von Krankheiten ist ausreichende Ernährung wichtig.“ \footcite{Blanckenburg1987ZukunftDie}
\end{displayquote}

%von Spebs Beginn
Laut der deutschen Gesellschaft für Ernährung benötigt ein erwachsener Mann im Durchschnitt 2300 kcal, eine erwachsene Frau 1800 kcal durchschnittlich, beide im Alter von 25-50 Jahren bei geringer körperlicher Aktivität. Das Essverhalten eines Menschen könne aber dennoch stark variieren.\footcite[Vgl.]{DeutscheGesellschaftfurErnahrunge.V.2015WieMensch} 
%von Spebs Ende

Aufgrund der steigenden Anzahl der Weltbevölkerung und der sich verändernden globalen Essgewohnheiten steigt der Bedarf an Anbaufläche, um diesen Lebensmittelbedarf zu decken. Vor allem in Entwicklungsländern und deren städtischen Regionen werden mehr tierische Produkte (d.h. Fleisch- und Milchprodukte), Zucker, Pflanzenöle und verarbeitete Lebensmittel konsumiert als in früheren Jahren. \\

Im Allgemeinen beschreibt der Begriff Lebensmittelversorgung im wirtschaftlichen Kontext die Bereitstellung von Lebensmitteln für private Haushalte.\footcite{UmweltbelastungenUmweltbundesamt}




\subsection{Nachhaltigkeit}
Da im Rahmen dieser Forschungsarbeit die Frage diskutiert wird, ob UL eine Form der nachhaltigen Lebensmittelversorgung bieten kann, muss zuerst einmal der Begriff Nachhaltigkeit mit seinen Facetten definiert werden, sowie das System in dem Nachhaltigkeit bestrebt wird.\\
Darauffolgend werden Indikatoren und Kriterien aufgelistet, welche hilfreich sind Produkte mit ihrem jeweiligen Herstellungsprozess zu bewerten. Im Anschluss werden Strategien vorgestellt, die sich dazu eignen Nachhaltigkeit umzusetzen.
Zum Abschluss wird der Begriff „nachhaltige Entwicklung” beleuchtet, welcher das Ziel dieser Methoden, Kriterien und Strategien darstellt. Siehe Abbildung \ref{fig:Folgerung_NE}.
\\
\\
Der Begriff Nachhaltigkeit ist vielschichtig geprägt. Er findet Verwendung in der Wirtschaft, Ökonomie, Ethik und Ökologie. Nachhaltigkeit kann als Art und Weise des Wirtschaftens bezeichnet werden, „bei welcher derzeitige Bedürfnisse befriedigt werden, ohne zukünftigen Generationen die Lebensgrundlagen zu entziehen.“ \footcite{DefinitionWirtschaftslexikonb}. Ebenso kann der Begriff als Brücke zwischen ökologischen und ökonomischen Interessen verstanden werden. Seinen Ursprung hat das Prinzip der Nachhaltigkeit in der Forstwirtschaft des 18. Jahrhunderts. Nach einer Übernutzung der Wälder und daraus resultierend knapper werdenden Holzbeständen, wurde mit Nachhaltigkeit ein Bewirtschaftungsprinzip gefordert, bei dem regenerativ gearbeitet werden sollte, das heißt “nicht mehr Holz geschlagen werden als nachwächst.“ \footcite{NachhaltigeBrockhaus.de}
Ab dem 19. Jahrhundert wurde dieser rein ressourcenökonomischen Betrachtungsweise von Nachhaltigkeit eine umfassendere hinzugefügt, die sämtliche Funktionen des Waldes in Betracht zieht, wie beispielsweise die Aufrechterhaltung seiner Schutzfunktion.\footnote{Schutzfunktionen des Waldes sind unter anderem Bodenschutz, Klimaschutz, Emmisions}
Es lässt sich zusammenfassen, dass Nachhaltigkeit als eine Haltung oder eine Verhaltensform bezeichnet werden kann, die sich zum Ziel gesetzt hat eine "Nachhaltige Entwicklung" (Siehe Kapitel \ref{NE} des Planeten Erde zu erwirken.

\hfill \break
Der Begriff Nachhaltigkeit kann in „stark” und „schwach” eingeteilt werden.\footcite{Nachhaltigkeit}
\begin{itemize}
\item Die starke Nachhaltigkeit stellt die Erhaltung der natürlichen Ressourcen in den Vordergrund. Sie beruht auf der Annahme, dass Naturgut nicht durch andere Kapitalformen ersetzt werden kann.
\item Schwache Nachhaltigkeit beruht auf der Annahme das Kapital- oder Naturgut durch andere Kapitalformen ersetzt werden kann.
\end{itemize}
Konfliktpotenzial zwischen den Vertretern jeweiliger Positionen tritt vor allem bei der Frage auf, „wie heute verursachte, aber zukünftig auftretende Umweltschäden beziehungsweise Ressourcenknappheiten zu bewerten sind“.\footcite{NachhaltigeBrockhaus.de}\\
\\
\textbf{Ökosystem}\\
Als Ökosystem bezeichnet man ein Wirkungsgefüge zwischen Lebewesen und ihrem Lebensraum. Einzelne Vorgänge innerhalb des Systems werden von Produzenten und Destruenten durchgeführt. Vereinfacht dargestellt entsteht auf diese Weise eine Nahrungskette. Dieses Nebeneinander aus verschiedenen Organismen befindet sich stets in einem zyklischen Prozess. Je enger die Nahrungsketten ineinander verwoben sind und je artenreicher das Wirkungsgefüge ist, desto komplexer und resilienter ist das Ökosystem.\footcite{NachhaltigeBrockhaus.de} Die Agrarlandschaft kann als ein vom Menschen stark beeinflusstes und künstliches Ökosystem angesehen werden.\footcite[Vgl.]{BrockhausOkosystem} Die Stoffkreisläufe innerhalb des Systems sind unter optimalen Umständen ausgeglichen. So entsteht ein dynamisches Gleichgewicht. \footcite{BrockhausOkosystem}

\hfill \break
Komplexe Ökosysteme sind stabiler als einfache Ökosysteme, gleichzeitig sind diese schwerer oder unmöglich regenerierbar, während einfache Ökosysteme sich und ihr Gleichgewicht einfach wiederherstellen können. Als Beispiel kann der Vergleich zwischen einem einfachem Ökosystem (Nadelwald) und einem komplexem Ökosystem (Regenwald) herangezogen werden.
In einem einfachen System kann der Verlust oder die Veränderung einzelner Komponenten (Überdüngung, Wasserknappheit oder Überwässerung in Brachregionen) das eigene Gleichgewicht sowie das Gleichgewicht angrenzender Ökosysteme schnell in Gefahr bringen. \footcite{DefinitionWirtschaftslexikone} Ein komplexes Ökosystem ist im Vergleich zwar resilienter und nicht so schnell aus dem Gleichgewicht zu bringen, werden allerdings einzelne Komponenten irreparabel geschädigt, wirkt sich das durch seine gegenseitige Abhängigkeit ebenso auf das gesamte System aus.
Die Erde kann als globales Ökosystem angesehen, welches sich aus Boden, Wasser und Luft und den darin lebenden Organismen zusammensetzt.Ebenfalls können Städte als Ökosysteme beschrieben werden. Sogenannte Stadtökosysteme sind Ökosysteme, die von Menschen erzeugt und beeinflusst werden. Wesentliche Merkmale eines Stadtökosystems sind der hohe Anteil an bebauter und versiegelter Fläche und eine hohe Bevölkerungsdichte. Diese hohe Dichte an verschiedenen Landnutzungen führt dazu, dass natürliche Ressourcen wie Boden, Wasser, Luft und Biodiversität extrem beansprucht werden. Desweiteren sei sein Erhalt Breuste et. al. zufolge von der externen Einfuhr an Lebensmitteln und Ressourcen abhängig.\footcite[S.61]{Breuste2016Stadtokosysteme}
\\
\\
%\begin{figure}[htbp]
%\centering
%\includegraphics[width=10cm]{image_folder/oekosystemkreisslauf.png}
%\caption{Ökosystemkreisslauf: Verknüpfung von Konsumenten Destruenten und Produzenten}
%\label{fig:Ökosystemkreisslauf}
%\end{figure}

\textbf{Ökologische Nachhaltigkeit}\\
„Die ökologische Nachhaltigkeit bezieht sich allgemein auf das Überleben und den Gesundheitszustand von Ökosystemen.“ \footcite{DefinitionWirtschaftslexikonc} Sie bezeichnet einen weitsichtigeren und rücksichtsvolleren Umgang mit natürlichen Ressourcen. Wenn die ökologische Nachhaltigkeit vernachlässigt wird, kann dies dazu führen, dass bestimmte Ressourcen unbrauchbar oder unwiderruflich zerstört werden, was jegliche weitere Entwicklung der Ressource und ihrer Umgebung unmöglich machen kann. Laut Berding und Bukow in „Die kompakte Stadt der Zukunft - Auf dem Weg zu einer inklusiven und nachhaltigen Stadtgesellschaft“\footcite[S.95]{BerdingWolf-DietrichBukowKarinCudakHrsgDieStadtgesellschaft} gewinnt das Thema Nachhaltigkeit für die Stadtentwicklung an neuer Relevanz. 



\subsubsection{Nachhaltige Entwicklung} \label{NE}
 Ziel der nachhaltigen Entwicklung ist eine dauerhalfte und gerechte Bewirtschaftung der Erde.\footcite{NachhaltigeBrockhaus.de} Die Bezeichnung wurde vom Begriff der Nachhaltigkeit abgeleitet. In der Agrar- und Ernährungswirtschaft definiert der Begriff eine „dauerhafte Nutzung von Ressourcen bei gleichbleibender bzw. wachsender Effektivität“. \footcite{oppenhauser2010nachhaltigkeit} Im internationalen Sprachraum hat sich der Begriff „Sustainable Development” als Definition gefestigt. In der internationalen Politik und bei gesellschaftlichen Bewegungen wird die nachhaltige Entwicklung als Leitbild eingesetzt.\footcite{oppenhauser2010nachhaltigkeit}
 
 Konkrete Umsetzungsmethoden von nachhaltiger Entwicklung stellen in Entwicklungsländern hauptsächlich Aspekte hinsichtlich des ökonomischen und sozialen Fortschritts in den Vordergrund. In den industrialisierten Ländern geht es dabei mehr um den langfristigen Schutz und Erhalt der natürlichen Lebensgrundlage.\\
\\
 Zur nachhaltigen Umsetzung von Produktionsmethoden werden ökoeffiziente Strategien und Konzepte angewendet. Die Kriterien für diese Strategien sind laut WBCSD folgende:
\begin{itemize}
\item Geringer Einsatz natürlicher Ressourcen
\item Geringe Umweltbelastung
\item Vergleichsweise hoher Ernteertrag
\end{itemize}

\textbf{Öko-Effizienz und Ressourceneffizienz im Vergleich}\\
Öko-Effizienz und Ressourceneffizienz geben beide eine relative Kennzahl an, die das Verhältnis einer gewünschten Größe zu ihrer unerwünschten Schadschöpfung\footnote{Schadschöpfung bezeichnet den ökologischen Schaden eines Produkts oder Prozesses der durch Emissionen oder Ressourcenverbrauch bei der Herstellung entsteht} darstellt. Bei beiden Begriffen wird eine  Maximierung des Betrags als positiv bewertet. Der Fokus ist bei den Begriffen jeweils ein anderer: 
\\
\\
Die \textbf{Öko-Effizienz} beschreibt das Verhältnis von ökologischer Schadschöpfung einer Tätigkeit oder eines Produkts zu seiner ökonomischer Wertschöpfung. \footcite[Vgl.][S.]{Schaltegger1990OkologischeUnternehmung} 
\\
 Der WBCSD\footnote{ein Zusammenschluss international tätiger Unternehmen, der das Ziel hat, Wirtschaftswachstum und Nachhaltigkeit in Einklang zu bringen} hat für Ökoeffizienz folgende Kriterien erstellt: 
 \begin{itemize}
 \item Die Bereitstellung wettbewerbsfähiger Produkte
 \item Die Befriedigung menschlicher Bedürfnisse 
 \item Die Förderung der Lebensqualität
 \item Die Minimierung der Umweltauswirkungen und Ressourcenintensität während des gesamten Produktlebens\footcite{OkoeffizienzBrockhaus.de}
 \end{itemize}
Die \textbf{Ressourceneffizienz} im Vergleich dazu legt den Fokus auf den Ressourcenverbrauch bzw. die Materialflüsse. Die Gleichung wäre hier also Nutzen im Verhältnis zum Aufwand oder auch Wertschöpfung im Verhältnis zum Ressourcenverbrauch.
\begin{figure}[htbp]
\centering
\includegraphics[width=10cm]{image_folder/ressourceneffizienz.jpg}
\caption{Ressourceneffizienzgleichung:}
\label{fig:Ressourceneffizienzgleichung}\footcite{Essel2010AnalyseFazit}
\end{figure}

   
\begin{figure}[htbp]
\centering
\includegraphics[width=10cm]{image_folder/NE_folgerungen.png}
\caption{Schlussfolgerungen aus Infos zu NE}
\label{fig:Folgerung_NE}
\end{figure}\footcite{Eigene Darstellung}


\subsubsection{Komplexität von Nachhaltiger Entwicklung}
Bei der Frage, wie nachhaltige Entwicklung effizient und lösungsorientiert umgesetzt werden solle, stellen sich neben der Problematik der Verantwortlichkeit für die Verursachung der Umweltprobleme und die Gerechtigkeitfrage nach einer gerechten Verteilung der endlichen Ressourcen („Global commons“). 
\begin{figure}[htbp]
\centering
\includegraphics[width=14cm]{image_folder/globalcommons.png}
\caption{3-dim-Modell}
\label{fig: global commons}
\end{figure}

Nach dem Umweltgipfel\footnote{(UN-Konferenz für Umwelt und Entwicklung (UNCED, Erd-, Rio- oder Umweltgipfel) Nach diesem Gipfel gewann das Thema Nachhaltigkeit erstmals an weltweiter Relevanz). Beteiligt waren 178 Staaten. Zwar wurden auf dem Gipfel einige langrfistige Dokumente, wie die Klimarahmenkonvention, die Walderklärung, die Rio-Deklaration und die Agenda 21 unterzeichnet, so enthält keines der Dokumente überprüfbare Verpflichtungen. Dennoch ging von dieser Konferenz ein Impuls aus und der Begriff nachhaltigen Entwicklung („Sustainable Development”) ist seitdem Bestand von politischen Programmen.} im Jahr 1992 in Rio de Janeiro wurde diese Frage „von zahlreichen Kommentatoren so beantwortet, dass [...] jeder Mensch weltweit das gleiche Recht hat, die globalen Gemeinschaftsgüter in nachhaltiger Weise zu nutzen. Dieser Interpretation wird entgegengehalten, dass sie regional unterschiedliche Bedürftigkeiten und kulturelle Besonderheiten ignoriere.” \footcite{NachhaltigeBrockhaus.de} Ebenso besteht der „Einwand, dass neben den unterschiedlichen Bedürftigkeiten auch das unterschiedliche Leistungsvermögen berücksichtigt werden müsse.“\footcite{NachhaltigeBrockhaus.de} Demnach sei das zulässige Nutzungsniveau eines Staates nicht nach der Größe seiner Bevölkerung, sondern nach seinem Beitrag zur globalen Wertschöpfung zu bestimmen.\footcite{NachhaltigeBrockhaus.de}

\begin{figure}[htbp]
\centering
\includegraphics[width=10cm]{image_folder/dreidimensionenmodell_der_N.png}
\caption{3-dim-Modell}
\label{fig:3-dimensionen Modell}
\end{figure}

Konfliktpotenzial besteht ebenfalls bei der Priorisierung von Lösungsvorschlägen. Eine nachhaltige Lösung im sozialen Sinn kann sich auf Kosten von ökologischer Nachhaltigkeit auswirken und umgekehrt. Genauso kann auch eine Win-Win Situation entstehen. Ein Beispiel wäre hier ein urbanes Projekt zum Anbau von Gemüse, welches gleichzeitig die Stadtgemeinschaft fördert und der Verkauf des Projekts genug Einnahmen erbringt, dass die Gemeinde sich im folgenden Jahr weitere Arbeitsutensilien kaufen kann.

Abbildung \label{fig:4-dimensionen Modell} zeigt die vier Dimensionen (sozial, ökologisch, ökonomisch und kulturell) der nachhaltigen Entwicklung: 

Entwicklungsländer stellen bei der Umsetzung bislang die sozialen und ökonomischen Aspekte in den Vordergrund. Da Industrienationen den Fokus auf die ökologische Dimension legen wird ihnen die Hauptlast bei der Lösung der bestehenden Probleme im ökologischen Bereich zugeschrieben. Industrieländer setzen ökologische Aspekte in den Vordergrund, nicht zuletzt weil sie es sich finanziell leisten können. Da eine ökologisch nachhaltige Lebensweise aller Industrienationen, die Umweltschäden, welche in Schwellenländern entstehen nicht wettmachen können, und diese einen großen Anteil der weltweiten Bevölkerung und Fläche ausmachen werden gleichzeitig Lösungsinitativen von Seiten der Entwicklungsländer gefordert.

\begin{figure}[h]
\centering
\includegraphics[width=6cm]{image_folder/vierdimensionenmodell_der_N.png}
\caption{Das Bild zeigt die vier Dimensionen der nachhaltigen Entwicklung: soziale Dimension, ökologische Dimension, ökonomische Dimension und kulturelle Dimension}
\label{fig:4-dimensionen Modell}
\end{figure}


\subsubsection{Bewertungsmethoden von Nachhaltigkeit}
Es gibt eine Vielzahl an Werkzeugen zur Bewertung von Nachhaltigkeit. Sie unterscheiden sich je nach Zielsetzung und Fokus. Neben vielen weiteren Methoden hat das Darmstädter Institut IINAS eine Reihe an Methoden als Datenbanken veröffentlicht. Diese sind Folgende:

\begin{itemize}
\item Ökoeffizienz-Analyse der BASF ( diese misst ökologische und ökonomische Dimension) und SEEBALANCE® (ökologische, ökonomische und soziale Dimension)
\item GEMIS - ein „Lebensweg- und Stoffstromanalyse-Modell mit integrierter Datenbank für Energie-, Stoff- und Verkehrssysteme”\footcite{}
\item GLOBALANDS: Entwicklung eines globalen Nachhaltigkeitsstandards zur Landnutzung
\item weitere Nachhaltigkeitsstrategien
\end{itemize}

\textbf{Bewertung von Nachhaltiger Landnutzung}
Das „Framework for Evaluating Sustainable Land Management” wurde im Jahr 1993 von der \acs{fao} (Ernährungs- und Landwirtschaftsorganisation der Vereinten Nationen) offiziell entwickelt und von Drechsel et al. im Jahr erweitert.

\begin{enumerate}
    \item „Die langfristige Produktivität des Bodens muss erhalten werden“.
    \item Das Risiko für die im Anbau Beschäftigten muss minimiert werden, inklusive das Risiko der Wegweisung von der bearbeteten Fläche.
    Natürliche Ressourcen müssen geschützt und erhalten bleiben, wobei Boden und Waser eine bbesondere Bedeutung zukommt. Zusätzlich muss der Gesundheit Dritter Rechnung getragen werden.
    \item Der Anbau muss wirtschaftlich rentabel sein um sein langfrisitges Überleben zu sichern.
    \item Die Auswirkungen des Anbaus müssen gesellschaftlich und politisch akzeptiert sein.”
\end{enumerate}

\textbf{Ökologischer Fußabdruck}\\
Der ökologische Fußabdruck ist ein Werkzeug zur Berechnung des menschlichen Naturverbrauchs und wird in globalen Hektaren angegeben.\footcite[S. 25]{MathisWackernagelUnserNimmt} Weltweit ist der ökologischer Fußabdruck einer der erfolgreichsten Indikatoren zur Vermittlung der „physischen Begrenztheit des Planeten Erde”.\footcite[S.2]{StefanGiljum2007WissenschaftlicheFuabdruck}
 „Der ökologische Fußabdruck misst so die ökologische Tragfähigkeit einer Bevölkerung“.\footcites[S.5]{MichelsenGrundlagenEntwicklung}[Vgl.][S.23ff]{MathisWackernagelUnserNimmt} Er berechnet dabei die  (Natur-)Fläche, die zur Aufrechterhaltung der Energie- und Materialflüsse einer Wirtschaftseinheit wie beispielsweise einer Stadt benötigt werden. 
„Die Inanspruchnahme von Ressourcen einer Person, einer Stadt oder eines Landes“\footcite[S.192]{AntjeFlade2015StadtStadtforschung} können somit anhand des ökologischen Fußabdrucks festgestellt werden. 

\begin{displayquote}
„Der ökologische Fußabdruck von London – als Beispiel für eine Großstadt in einem hoch entwickelten Land – ist z.B. um etwa das Dreihundertfache größer als die Stadtfläche von London (Wackernagel et al. 2006), und damit wird deutlich, dass die Bevölkerung Londons auf ein Vielfaches der Stadtfläche – nicht nur im Umland, sondern weltweit – angewiesen ist. Da die globale Menge an Biokapazität begrenzt ist, veranschaulicht der ÖF ebenso das bestehende Ungleichgewicht.”\footcite{AntjeFlade2015StadtStadtforschung, S.192}
\end{displayquote} 

Durch die Berechnung dieses Fußabdrucks kann also nachvollzogen werden ob ein jeweiliger Konsum nachhaltig ist oder nicht.\footcite[S.192]{AntjeFlade2015StadtStadtforschung}

Kritik am ökolgischen Fußabdruck besteht unter anderem, da der „Fußabdruck die wichtige Dimension der nicht-erneuerbaren Ressourcen nur indirekt einbezieht.”\footcite[S.3]{StefanGiljum2007WissenschaftlicheFuabdruck} So gebe der Fußabdruck lediglich vor, die Grenze von Nachhaltiger Nutzung vorzuweisen, dabei würden die Berechnungen auf zum Teil fraglichen Annahmen basieren. \footcite[Vgl.][S.3f]{StefanGiljum2007WissenschaftlicheFuabdruck}
\\
\\
\begin{figure}[htbp]
\centering
\includegraphics[width=12cm]{image_folder/oekFussabdruckWelt.png}
\caption{der ökologische Fußabdruck der Kontinente im Vergleich}
\label{fig:oekFussabdruckWelt}
\end{figure}

\begin{figure}[htbp]
\centering
\includegraphics[width=12cm]{image_folder/oekFussabdruckZeit.png}
\caption{der ökologische Fußabdruck je nach Verwendung und Zeit}
\label{fig: oekFussabdruckZeit}
\end{figure}

Schlussendlich stellen die zuvor genannten Bewertungsstrategien lediglich eine Annäherung oder einen Versuch zur Quantifizierung des Ressourcenverbrauchs von Prozessen dar. Der Unterschied besteht in der jeweiligen Fokussierung auf gewissen Parameter.  Als feste Berechnung können sie dabei nicht gewertet werden. Jedes der Konzepte weist auf seine Weise Lücken und Schwachstellen auf. 




Um Projekte hinsichtlich ihrer Nachhaltigkeit ganzheitlich zu beurteilen, bedarf es viel Zeit und Recherche. Gleichzeitig muss bei der Untersuchung entschieden werden welche Dimensionen der Nachhaltigkeit primär betrachtet werden soll, da wie bereits genannt, Projekte je nachdem mit welchem Fokus sie betrachtet werden unterschiedlich ausfallen.\\
\\
Im Rahmen dieser Forschungsarbeit werden Methoden bzw. Aspekte der \acs{ul} vorgestellt und hinsichtlich ihres Betrags zur nachhaltigen Entwicklung analysiert. Bei der Bewertung soll die ökologischen Dimension der nachhaltiger Entwicklung im Vordergrund stehen. Aspekte der ökonomischen Dimension können unter Umständen mit einfließen, da diese Faktoren sich häufig gegenseitig bedingen. Soziale und kulturelle Kriterien werden in dieser Forschungsarbeit nicht betrachtet, da dies den Rahmen der Forschungsarbeit sprengen würde.

\subsubsection{Ausgewählte Nachhaltigkeitskriterien} \label{UnsereKriterien}

Aus den zuvor genannten Methoden zur Bewertung von Nachhaltigkeit wurden folgende Kriterien zur Orientierung ausgewählt, hinsichtlich welcher Beispiel-Projekte bewertet werden sollen.

\begin{itemize}
\item Schutz natürlicher Ressourcen (wie Boden, Wasser oder Luftqualität)
\item Nutzung regenerativer Energie
\item Sparsamer Ressourcenumgang (von Anbau bis Konsum)
\item Schadstoffreduktion ( z.B. Emissionen, Abfall oder Pestizide)
\item Kreislaufsystem
\item Förderung der Biodiversität
\item Rücksicht Dritter
\item Bedachte Regenerationszeit der Natur
\item Ökonomische Wirtschaftlichkeit
\item Langfristigkeit
\item Gesellschaftliche und politische Akzeptanz
\end{itemize}


\subsubsection{Kritik am Begriff der Nachhaltigkeit}\footcite{NachhaltigeBrockhaus.de}
Die übermäßige Verwendung des Begriff Nachhaltigkeit in der Wirtschaft und Politik hat bei vielen Bürgern zu Misstrauen geführt. Kritisiert wird vor allem, dass der Begriff überladen sei und als Sammelbegriff für alles Gute genannt werde, was unerfüllbare Erwartungen wecke. Gleichzeitig sei der Begriff beliebig geworden: “Unter der Flagge der nachhaltigen Entwicklung könne man trotzdem für komplett gegensätzliche Dinge eintreten.” So entstände die Meinung, Nachhaltigkeit sei inhaltslos und habe nur rhetorische Funktionen, was die Leitbildfähigkeit des Begriffs außer Kraft setze. Noch schärfer sind Stimmen, die Nachhaltigkeit von vornherein als Utopie oder Illusion bezeichnen.
\\
Es darf geschlussfolgert werden, dass aufgrund dieser Unsicherheit hinsichtlich des eigenen Verhaltens und der zukünftigen Entwicklung der Umwelt eine große Frustration bei der Bevölkerung entstanden ist. Und es entsteht die Frage, welche Lösungsansätze zu einer nachhaltigen Entwicklung der Welt beitragen. 

\begin{figure}[h]
\centering
\includegraphics[width=5cm]{image_folder/skizze1.jpg}
\caption{Nachhaltigkeit der LV}
\label{fig:Skizze_Nachhaltigkeit}
\end{figure}

\FloatBarrier

\section{Die Distanz zwischen Stadtbevölkerung und Landwirtschaft}
\subsection{Verstädterung im Hinblick auf die Lebensmittelversorgung} \label{Vergangenheit der Urbanen Landwirtschaft}
Wie Städte entstanden sind und sich bis heute ausbreiten hat viel damit zu tun, wie Menschen das Problem der Nahrungsmittelsicherung lösten. Elmquist, Redman, Barthel und Costanza beschreiben in „Urbanization, Biodiversity and Ecosystem Services: Challenges and Opportunities“ hierzu drei vergangene Lösungsansätze, an dem sich dieses Kapitel anlehnt.\footcite[S.14ff]{Elmqvist2013} 
\\
\\ 
Der erste Ansatz war das Nomadentum. Menschen zogen von Gegend zu Gegend, um sich zu ernähren. Sie folgten Tierherden sowie klimatisch günstigeren Regionen, wo nahrhafte Rohstoffe zu finden waren. Der zweite Ansatz war die Domestizierung von Tieren und Pflanzen. D.h. Rohstoffe wurden passend nach Bedarf herangezüchtet, geerntet und wieder angebaut. Menschen waren dadurch weniger wechselnden Bedingungen ausgesetzt und verbesserten stetig ihre Nahrungsmittelerzeugung. Das erlaubte dauerhafte Sesshaftigkeit und förderte eine sichere Existenzgrundlage. Aus diesem Grund setzten sich kleine ländliche Gemeinden zur verbreitetesten Siedlungsform auf der Erde durch. Sie waren zwar unabhängig und isoliert voneinander, zeichneten sich allerdings als besonders widerstandsfähig gegenüber dem vorherigen Lebensstil aus. Für Elmquist et. al. war es also keine Überraschung, dass die Bevölkerungszahl dadurch anstieg. Darüber hinaus wuchsen ländliche Gemeinden derart, dass die Menge an Menschen überschüssig zur nötigen Arbeit auf dem Acker war.\\
\\
Den Höhepunkt dieses Prozesses bildete 5500 v.C. die erste städtische Siedlung in Mesopotamien - der dritte Ansatz. Die dortigen Innovationen wie das Schreiben, Monumentalbauten oder vertiefende Handwerkskünste zeigten das Potenzial dieser neuen Siedlungsform. Die Autoren bezeichnen Städte als „Zentren der Innovation“. Verglichen mit ländlichen Gemeinden zeichnet sich eine Stadt durch eine äußerst diverse und voneinander abhängige Gemeinschaft aus. Ländliche Gemeinden ist zwar unabhängiger und selbstversorgend, dafür ist das Wachstum dieser Siedlungsform begrenzt. Anders verhält es sich mit Städten. Bezüglich der Versorgung war eine Stadt auf die umliegenden Gemeinden angewiesen. Jedoch trug der Austausch von Waren, technologischen Erfindungen sowie wissenschaftliche Erneuerungen enorm zum Wachstum und der Verbreitung von Städten bei. Hinzu kam, dass eine soziale Erneuerung nötig war um diese große Gruppe von Menschen und groß skalierte produktive Aktivitäten zu organisieren. Eine klassenstrukturierte Gesellschaft, eine formalisierte Rechtsgrundlage und eine territoriumsbasierte Regierung bildeten daher den charakteristischen Rahmen. Damit eine Gemeinschaft auf diesem dichten Lebensraum funktionierte, war nämlich eine gewisse Ordnung im Miteinander erforderlich. Diese Siedlungsform erwies sich so erfolgreich, dass viele Städte Wohlstand und Sicherheit genossen. 
\\
\\
Der wachsende Wohlstand in Städten führte zu regelmäßigen militärischen Angriffen. Zum Schutz wurden daher Stadtmauern gebaut. Diese Sicherheitsmaßnahme wiederum förderte die Produktivität und damit den Wohlstand. Das wiederum führte zu einer wachsenden Bevölkerung und durch die Eingrenzung zu einem dichteren Wohngefüge, das kennzeichnend für viele heutige Städte ist. Der erfoderliche Anbau an Rohstoffen fand daher meistens im Umland statt und weitere Arbeitskräfte auf dem Land siedelten nahe der Stadt. \\
\\
Trotz der Tatsache, dass Städte unterschiedlich entstanden sind, sehen Elmquist et. al. folgende Gemeinsamkeit:\footcite[Vgl.][S.19ff]{Elmqvist2013} Die Anhäufung von Menschen in einer Stadt hatte Spezialisierung von diversen Beschäftigungen der Bewohner zur Folge. Dies führte dazu, dass der Großteil der urbanen Bevölkerung immer weniger selbstversorgende Tätigkeiten ausübte. Dadurch wurde einem großen Anteil der ruralen Bevölkerung die Verantwortung übertragen so viele Nahrungsmittel zu produzieren, dass sie sowohl sich selbst als auch die Stadtbevölkerung versorgen konnten und gleichzeitig ausreichend Gewinn machten um die Verteilung und den Transport dieser Güter zu decken. Sie übernahmen demnach die Rolle des Versorgens in diesem Stadt-Land-System. Das war eine große Last im Vergleich zur blosen alleinigen Versorgung. Auch als die sozialen Rollen der urbanen und ruralen Bevölkerung auseinanderdrifteten änderten sich die jeweiligen Ziele und Verständnisse zur Landwirtschaft. Dementsprechend ging es um die maximale Produktion auf Kosten der regenerativen Kapazitäten der natürlichen Ressourcen. Elmquist spricht hier von  „ [...] maximizing short-term returns with little concern for long-term consequences.“\footcite[S.20]{Elmqvist2013}\\

\begin{figure}[h]
\centering
\includegraphics[width=12cm]{image_folder/connections_1.jpg}
\caption{Früher war die Versorgung der Städte abhängig vom Umland, mit einem direkten Austausch. Heute hingegen hat sich eine globale Abhängigkeit mit komplexen Lieferketten entwickelt.}
\label{fig:verbindungen}
\end{figure}

\\
Zusammenfassend lässt sich sagen, dass Städte bereits seit ihrer Existenz eine abhängige Beziehung zu ländlichen Gemeinden haben. Denn der Überschuss in der Landwirtschaft sorgte erst dafür, dass diese Siedlungsform entstehen konnte. Städte sind zwar globale bzw. regionale Zentren der Innovation und Administration, waren aber seit Anbeginn angewiesen auf die Einfuhr von Ressourcen. Gleichzeitig übernahm die rurale Bevölkerung die immer größer werdende Verantwortung der Lebensmittelversorgung. 

\subsection{Entwicklung globaler Ernährungssysteme} \label{entwicklungglobal}
In diesem Kapitel wird die Gewichtung von lokalen Städten auf die globale Perspektive zur Lebensmittelversorgung gewechselt, um ein tieferes Verständnis über die beiden Komponenten zu erhalten.
\\
\\
Wie es zur Entwicklung der aktuellen globalen Lebensmittelversorgung gekommen ist, teilen Friedmann und McMichael in drei Phasen auf. Diese Phasen bezeichnen sie als sogenannte Lebensmittelregime, ein Konzeptrahmen das zur Untersuchung dient welche internationalen Verbindungen zur Lebensmittelproduktion im Zeitraum kapitalistischer Transformation vorherrschten.\footcite[Vgl.][S.95]{Friedmann1989AGRICULTUREPresent}

\subsubsection*{Anfänge nationalen Handels}
Das erste Lebensmittelregime umfasste die Zeit 1870 - 1914. Das sei einerseits der Höhepunkt der Kolonisation und andererseits die Entwicklung von Nationalstaaten. Die besiedelten Kolonien (wie Australien, Kanada, Neu Seeland und die Vereinigten Staaten), europäische Kolonialmächte und die beherrschten Kolonien (wie zum Beispiel Indien) waren demnach die maßgebenden Akteure in dieser Zeit. Durch neue Technologien war das Ernährungssystem über mehrere Kontinente erst möglich. Züge gewährleisteten den Transport großer Mengen an Rohstoffen. Wesentlich in dieser Zeit sei der Import von Getreide und Fleisch nach Europa und im Gegenzug die Lieferung von hergestellten Waren, Arbeit und Kapital. Der Anreiz nach Beschäftigung ließ viele europäische Familien in diese Siedlerstaaten ziehen. Das förderte die verlagerte landwirtschaftliche Produktion in den Siedlerstaaten, was sowohl dem Bedarf an Nahrungsmitteln und dem Platzmangel in den europäischen Staaten in mehreren Gesichtspunkten entgegen kam. Auch durch das günstige Klima und die billige Arbeitskraft konnten sie landwirtschaftliche Produkte preiswerter herstellen, als es im eigenen Land möglich war. Desweiteren optimierten landwirtschaftliche Erfindungen wie Dünger und Maschinen die Ernte. Die Handelsbeziehungen zwischen den Siedlerstaaten und Europa sei die Basis eines ersten internationalen Handelssystems. Friedmann et. al. zufolge führten die Importe von Getreide und Fleisch von Siedlerstaaten und der Export von Kapital und Menschen um jene Lebensmittelproduktionen zu organisieren zum industriellen Kapitalismus.\footcite[Vgl.][S.96ff]{Friedmann1989AGRICULTUREPresent}


\subsubsection*{Nationaler Austausch von Gütern}
Friedmann et. al. zweites Regime beinhalte das Ende des zweiten Weltkrieges bis etwa 1970. Den Autoren zufolge waren nach dem zweiten Weltkrieg die Vereinigten Staaten von politischem Interesse getrieben, ihren Export an Lebensmitteln auszubauen. Im Anbruch des kalten Krieges wollten sie die Industrialisierung in Entwicklungsländern und in Europa vorantreiben. Durch die Subvention ihrer heimischen Landwirtschaft sorgten sie zu Produktionsüberschüssen von Getreide und Mais, was sie zum einen als Hilfspaket an europäische Staaten schickten und preiswert an Entwicklungsländer anboten. Gleichzeitig beschränkten sie den Import von ausländischem Getreide. Das hatte zur Folge, dass in den Entwicklungsländern eigene Lebensmittel verdrängt wurden und diese abhängig vom Export waren. Desweiteren wurde der Preis tropischer Rohstoffe aus diesen Ländern gering gehalten, was das Ungleichgewicht in dieser Handelbeziehung verstärkte. Ein weiterer wesentlicher Aspekt in dieser Zeit war der Wandel zur Massenproduktion in Form von intensiver Tierhaltung und der Herstellung haltbarer Lebensmittel in den sechziger Jahren. Landwirte in den Industriestaaten wechselten ihre Rolle vom Lebensmittelproduzent zum Lieferant eines Rohstoffes. Auch ihr Kundenkreis verlagerte sich vom Endverbraucher zu großen Firmen. Das Ende dieser Phase kündigte sich an als zum einen europäische Staaten sich zu starken Lebensmittelexporteuren entwickelten und dem amerikanischen Export Konkurrenz machte. Zum anderen stärkten Entwicklungsländer ihre heimische Agrarwirtschaft und schränkten Exporte ein. Dadurch waren Industriestaaten plötzlich angewiesen auf Importeure, die ihre Überschüsse abnahmen.\footcite[Vgl.][S.103ff]{Friedmann1989AGRICULTUREPresent}

\subsubsection*{Globalisierung der Lebensmittelproduktion}
Das dritte Lebensmittelregime sei geprägt von einer Globalisierung der Lebensmittelindustrie. Die Lebensmittelproduktion, ihre politische Förderung und Regulation, biochemische Innovationen, die standortunabhängige Produktionen ermöglichen als auch der Verkauf geht über nationale Grenzen hinweg. Stierrand spricht hier von der Bedeutung „transnationaler Firmen“ als auch die „Internationalisierung von Konsummustern und des Geschmacks verbreitet durch Massenmedien“\footcites[Vgl.][S.24ff]{Stierand2008StadtLebensmittel}, die sich in Angebot und Nachfrage ergänzen. Im Hinblick auf die Rolle der Landwirte äußert sich der Autor folgendermaßen:
\begin{displayquote}
„Mit der Veränderung des weltweiten Ernährungssystems haben sich die Schwerpunkte und die Machtstrukturen innerhalb des weltweiten Ernährungssystems verlagert. [...] Mit sinkendem Anteil der Landwirte an der Produktion von Lebensmitteln, gewannen andere Akteure größere Anteile der Wertschöpfung und Macht. Dabei hat der Profit in der Landwirtschaft noch nie einen wichtigen Anteil am nationalen Einkommen eingenommen. Ihre Funktion für die nationale Ökonomie war und ist die Bereitstellung von billigen Lebensmitteln, so dass der Anteil am Lohn, der nicht für weitergehenden Konsum zur Verfügung steht, möglichst gering ist.“\footcite[S.25]{Stierand2008StadtLebensmittel}
\end{displayquote}

Abschließend sind aus den Kapiteln 6.1. und 6.2. folgende Merkmale zu erkennen: Im allgemeinen lösten sich Menschen von der lokalen Selbstversorgung zur globalen Fremdversorgung. Die Distanz zwischen Stadtbewohnern - den Konsumenten - und Landwirten erhöhte sich im sozialen Sinne durch die Anreihung von weiteren Akteuren in der Produktionskette und physisch durch die vielfältige geographische Verteilung der Produktions- und Vertriebsstätten. Waren die versorgende Bevölkerung meistens noch im Umland von Städten positioniert, erschwert es die aktuelle Komplexität des Ernährungssystems diese nachzuvollziehen. Folgt man Stierrands obiger Aussage, erkennt man auch die Distanz zwischen Konsumenten und Landwirten anhand der wirtschaftlich geringen Wertschätzung. Diese Beziehung zur Landwirtschaft wirkt sich auch auf das Verhältnis zu Lebensmitteln aus, was in Kapitel \ref{StadtLebensmittel} näher erläutert wird. 

\section{Romantisierung der Urbanen Landwirtschaft}

\subsection{Aufkommen der „Landlust“ und deren Formen}

Während zuvor die wachsende Distanz zwischen Stadtbewohnern und Landwirtschaft erklärt wurde, geht dieses Kapitel auf die damit einhergehende Wahrnehmung dieser Distanz ein. Die Wahrnehmung der urbanen Landwirtschaft hängt häufig vom kulturellen Hintergrund ab. Während die grundsätzliche Versorgung von Lebensmitteln in Entwicklungsländern an erster Stelle steht, verlagert sich der Fokus von Industrieländern auf die Herkunft der Lebensmittel. Lebensmittel sollen immer häufiger möglichst natürlich und ohne künstliche Zusätze beziehungsweise ohne Chemische Wachstumsverstärker produziert werden, die Tierhaltung hingegen soll möglichst tierfreundlich und unter Freilandhaltung stattfinden. Daraus stellt sich die Frage, ob die Bevölkerung zum Beispiel Deutschland noch einen realistischen Blick auf ihre Lebensmittel besitzt? Oder ist das Bild welches die deutsche Bevölkerung von Nahrungsmitteln durch Medien und Vorstellungen erhält schon längst verschoben? Fest steht, dass derzeit jeder Deutsche mit seinem Konsumverhalten im Durchschnitt schon mehr Agrarfläche für den Anbau von Lebensmitteln und Konsumgütern benötigt als ihm rein rechnerisch zusteht. Diese Thematik wird in Kapitel 9 näher beschrieben. Im folgenden Absatz wird nicht beschrieben ob die Lebensmittelproduktion der hohen Nachfrage durch natürliche Anbaumethoden und Freilandhaltung nachgegangen werden kann. Vielmehr wird beschrieben, wie die Romantisierung der Lebensmittel in Deutschland aufkommt und wie sie sich auswirkt.\\
\\
Das Vorbild des folgenden Absatzes bildet zudem Egnolff, welche in \textit{Die Sehnsucht nach dem Ideal} näher auf die Thematik der „Landlust“ und des \acs{ug} eingeht. Eine romantische Vorstellung des Landlebens und der Landwirtschaft bei der Begrünung von Städten ist ein weit verbreitetes Phänomen und dazu nicht neu. Da Landwirtschaft bis zur Phase der Industrialisierung auch in der Stadt oder im städtischen Umland stattfand und die Versorgung so durch eigens angebaute Lebensmittel in der Subsitenzwirtschaft bewerkstelligt wurde, bildete diese ursprüngliche Form der Landwirtschaft in Deutschland und auch global gesehen die Tradition der Nahrungsmittelversorgung ab. Erst mit der Verdichtung der Städte und der Umwandlung von einer Agrar- zur Industriegesellschaft verschwanden die städtischen Gärten und Landwirtschaftsflächen. Dieser Gesellschaftsumschwung führte dann zur Urbanisierung. \footcite[Vgl.][S. 32ff]{Egnolff2015DieIdeal}\\
\\
Enge Wohnverhältnisse, schlechte Hygiene und niedrige Lebensstandards im Zuge dieser führten zur aufkommenden „Landlust“ und der Rückbesinnung auf ländliche Freiräume. Die „Landlust“ in Deutschland, welche in der folgenden Arbeit durch das Verlangen nach einem natürlichem und ländlichen Lebensstil definiert wird, äußerte sich dann beispielsweise in der Romantisierung des Landlebens und der Großstadtkritik außerdem im UG, im Konzept der Gartenstadt-, die Siedlungs- und die Kommunenbewegung sowie im Kleingartenwesen. Diese Bewegungen sollten einen Gegensatz zur Industriewirtschaft und der damaligen Politik vom letzten drittel des 19. Jahrhunderts bis zum ersten Weltkrieg bilden.\footcite[Vgl.][S. 35]{Egnolff2015DieIdeal}\\
\\
Hinter den Siedlungs- und Kommunenbewegungen stand die Idee einer Gemeinschaft die soziale und ökonomische Gleichheit sowie körperliche und geistige Gesundheit anstrebten. Hierzu diente das Landleben und die damit verbundene Rückbesinnung als Strebensziel. Die Romantisierung des Landlebens aus Sicht der Stadtbevölkerung als Raum für das gesunde Leben in Freiheit und Idylle ging aus dem Wunsch nach dem Gegenteil zum weniger natürlichen Stadtleben hervor. Diese Bestrebungen mündeten dann in die Gartenstadtbewegung welche die Misstände in der Stadtbevölkerung durch die Planung von Gartenstädten im Umkreis abfangen wollte. Die Gartenstädte sollten ein Zusammenspiel von Stadt und Land bieten welches die Lebensbedingungen der Einwohner verbessern sollte. Diese Idee geht wiederum auf Ebenezer Howard zurück welcher 1889 in seinem Werk „Garden Cities of To-morrow“ (ursprünglich „Tomorrow: A Peaceful Path to Real Reform“) dieses Konzept beschreibt. Die erhofften Entwicklungen fanden dennoch nicht im entsprechenden Ausmaß statt.\footcite[Vgl.][S. 36]{Egnolff2015DieIdeal}\\
\\
\begin{displayquote}
„Die tatsächliche Umsetzung der Gartenstadt fand jedoch in suburbanen Gebieten und nicht als Verknüpfung von Stadt und Land statt. Die erhofften gesellschaftlichen Veränderungen blieben aus und die Bilanz der Wohnungsstatistiken der Gartenstadt-Gesellschaften war insgesamt niedrig“\footcite[S. 36]{Egnolff2015DieIdeal} 
\end{displayquote}


Das Bundeskleingartengesetz definiert Kleingärten „zur nichterwerbsmäßigen, gärtnerischen Nutzung, insbesondere zur Gewinnung von Gartenbauerzeugnissen und zur Erholung“\footcite[§ 1]{Verbraucherschutz2006BundeskleingartengesetzBKleingG}

Das Kleingartenwesen entwickelte sich aus der Reaktion gegen beengte und unhygienische Wohnverhältnisse heraus. 

\begin{displayquote}
„Die Gärten sollten als Ausgleich zum Arbeitsalltag und zur Selbstversorgung dienen und den Bevölkerungsgruppen, die aus ländlichen Gegenden zuwanderten, die Eingliederung in die urbane Gesellschaft erleichtern. Zusätzlich sollten sie der Entfremdung des Menschen von der Natur, die mit der Industrialisierung
und Urbanisierung auftrat, entgegenwirken.“ \footcite[Vgl.][S. 38]{Egnolff2015DieIdeal}
\end{displayquote} 

Die Gründung des ersten Schrebergartens ging auf Ernst Innocenz Hauschild zurück der diesen als Spielplatzinitiative erdachte. \footcite[Vgl.][S. 39]{Egnolff2015DieIdeal}

Im Zuge der Urbanisierung ziehen die vermeintlichen Vorteile der Stadt nach wie vor Landbewohner in verdichtete Gebiete. Die Erwähnte Vorteile sind überweigend: 

\begin{itemize}
\item bessere Infrastruktur
\item mehr Abwechslung
\item mehr Erfahrungsmöglichkeiten 
\item mehr soziale Begegnungen
\item ein größeres Angebot von Arbeitsplätzen
\item mehr Konsummöglichkeiten
\end{itemize}

Die wahrgenommenen Nachteile sind oft die Belastungen im Zusammenhang mit dem Verkehr und die schwierige Wohnungssituation.\footcite[S. 41]{Egnolff2015DieIdeal} Trotz der Vorteile glaubt die Mehrheit der deutschen dass das Leben auf dem Land qualitativ hochwertiger sei. Dies liegt oft an einer übertriebenen Wahrnehmung für das ländliche Leben.

Eine 2014 durchgeführte Umfrage des Allensbach Institus für Demoskopie zur „Sehnsucht der Stadtbewohner nach Ländlichkeit“ belegte den Trend der Landlust. So sind 40\% der Befragten der Meinung, dass das Leben auf dem Land lebenswerter sei.\footcite[Vgl.][S. 15 Abb.2]{Dr.ThomasPetersen2014DieLandlichkeit}

Die Vorteile für das Land seien demnach:

\begin{itemize}
    \item Gesundheit
    \item Natürlichkeit
    \item gute Luft
    \item Nachbarschaftshilfe
    \item günstiger Wohnraum
\end{itemize}

Interessant ist, dass 27 \% der 1520 Befragten die Assoziation „einsam“ dem Landleben, aber zu 39 \% dem Leben in der Stadt zuordnen. Laut Dr. Petersen handele es sich hierbei um ein Klischee demnach Menschen in der anonymen Großstadt vereinsamen. Dieses Klischeebild hätte sich in der Umfrageforschung über Jahrzehnte hinweg jedoch nie bestätigen lassen. Dennoch halte sich das Klischee bis heute hartnäckig.\footcite[Vgl.][S. 7ff]{Dr.ThomasPetersen2014DieLandlichkeit}

\begin{figure}[htbp]
\centering
\includegraphics[width=11cm]{image_folder/SchaubildStadtVsLand_Umfrage.png}
\caption{Eigene Zeichnung, entstanden aus dem Vorbild von Dr. Thomas Petersen, Die Sehnsucht der Stadtbewohner nach Ländlichkeit S.13, Abb.3}
\label{fig:SchaubildStadtVsLandUmfrage}
\end{figure}

\begin{displayquote}
„Je mehr Menschen in der Stadt leben, je weniger Kontakt sie zum tatsächlichen Landleben haben, desto mehr wird das Land zu einer Projektionsfläche ihrer Phantasien.“\footcite[S. 8]{Dr.ThomasPetersen2014DieLandlichkeit}
\end{displayquote}

Die Kategorien Stadt und Land können die räumliche, soziale und wirtschaftliche Realität nach dem Stadtforscher Angelus Eisinger kaum mehr beschreiben. Er schlät staddessen den Begriff „Stadtland“ als schwer trennbare, ineinanderfließende Siedlung vor und betont, dass es sich bei „Land“ lediglich um eine Idealisierung von Vergangenem handele. \footcite[Vgl.][S. 40]{Egnolff2015DieIdeal}

Laut Agraringenieur und Geschäftsführer des AgrarBündnis e.V Dr. Frieder Thomas hätten Landwirtschaftsvertreter des AgrarBündnis einerseits Verständniss für die aus seiner Sicht unrealistischen Vorstellungen des Landlebens, könnten sie aber trotz bester Absichten nicht erfüllen. \footcite[Vgl.][S. 27]{Thomas2015BauerlichkeitBegriff}

\begin{displayquote}
„Die Wünsche und Bilder, die hinter diesen Trends stehen, haben mit der Realität der Landwirtschaft und dem Leben auf dem Lande oft wenig gemein. In diesem Punkt sind sich die Verfechter bäuerlicher Landwirtschaft und die Strategen der Wachstumslandwirtschaft ausnahmsweise einig. Beide leiden auf ihre Art unter den vielfältigen Hoffungen, die auf den Bauern und dem Landleben ruhen. Beide stehen mitten zwischen den Bildern der Bauernhofidylle und dem Wachstumsdruck.“ \footcite[S. 27]{Dr.Thomas2015BauerlichkeitBegriff}
\end{displayquote}

Andere Vertreter würden wiederum eine Doppelstrategie verfolgen, in der sie einerseits die unrealistischen Visionen als ideologisch und unwissenschaftlich bezeichnen andererseits aber das realitätsferne Bild der Landwirtschaft mit einem Image der Idylle innerhalb von Werbestrategien untermauern.\footcite[S. 27f]{Dr.Thomas2015BauerlichkeitBegriff}

Trotz der großen Distanz zur Realität bietet diese Sehnsucht nach einem besseren Leben auf dem Lande die Chance, die Art und Weise, wie wir Landwirtschaft betreiben und Lebensmittel handeln und behandeln, wieder zu einem zentralen Thema unserer Gesellschaft zu machen.

Es lässt sich feststellen, dass die Landlust besonders stark in städtischen Siedlungen ausgeprägt ist. Dies hängt zum Teil mit der fehlenden Ruhe und Idylle wie auch mit dem Bedürfnis nach freier Entfaltung in „gesunder Umgebung“ zusammen. Trotz der unterschiedlichen Ansichten von städtischer und landwirtschaftlicher Bevölkerung, die im Hinblick auf das Leben in ländlichen Gebieten bestehen, kann die Sehnsucht nach dem Leben auf dem Land oder dem „Ländlichen“ genutzt werden. So werden Stadtbewohner immer sensibler wenn es um das Thema „Leben auf dem Land“ und die Erzeugung von Nahrungsmitteln geht. So ist anzunehmen, dass das Interesse für diese Thematik steigt und es zu einem zentralen Thema unserer Gesellschaft wird. Für die erfolgreiche Etablierung von \acs{ul} gilt es jedoch im Umkehrschluss eine übertriebene Erwartungshaltung gegenüber der ländlichen oder der ruralen Landwirtschaft durch innovative \acs{ul}-Konzepte teilweise oder ganz zu befriedigen oder durch neue Ansätze zu ersetzen. So könnte UL auch zum Wohlbefinden der Stadtbevölkerung beitragen. Dies setzt jedoch vorraus dass die städtische Bevölkerung das teilweise übertriebene, rurale und romantische Bild der Nahrungsmittelproduktion ablegt und die Vorteile von {ul} auch dann anerkennt und wenn es sich bei ihr um automatisierte Indoor-Farmen bzw. Vertikale-Farmen handelt, die nicht dem gängigen Bild von Landwirtschaft entsprechen.\\
\\
Dennoch scheint die Landlust oft auch mit der Verschlechterung der Lebenssituation in Bezug auf Verschmutzung, Enge, fehlender Freiheiten und vor allem Nahrungsmittelknappheit zusammenzuhängen. Aus diesem Grund wird im folgenden Absatz der Ursprung der \acs{ul} unter dem Aspekt der Subsitenz näher beleuchtet.

\section{Ursprung der Urbanen Landwirtschaft}

Betrachtet man die Entwicklung und die bisherige Ausgestaltung von Städten, (wie in Kapitel \label{DieDistanzzwischenStadtbevölkerungundLandwirtschaft} beschrieben) kann man dahinter leicht ein bestimmtes System entdecken. Neben der eher geplanten Verortung der Städte fand eine klare Differenzierung zum ländlichen Raum statt. Städte oder Siedlungen wurden zwar, wie bereits erwähnt, in Gebieten gegründet, die Versorgung durch Nahrung sicherstellten, dennoch führte die urbane Verdichtung dazu dass Nahrungsmittel, immer häufiger aus dem Umland in die Städte befördert werden mussten. Hierbei stellt die Megametropole New York ein Beispiel dar. Die Entwicklung der Stadt zeigt durch ihre verdichtete urbane Struktur eindeutig eine Tendenz, die keine rurale Landwirtschaft zulässt.\footcite[Vgl.][S. 146]{MullerUrbanStadt} Die Stadt selbst fungiert hierbei als Lebensraum und infrastrukturelle Anlaufstelle nicht als Raum für die Landwirtschaft. Diese Form des Städtebaus überdauerte und existiert bereits bis in die Gegenwart.

\subsection{Frühe Pläne zur Urbanen Landwirtschaft}
Dabei existieren bereits seit mehr als 100 Jahren Stadtpläne durch ambitionierte Stadtplaner die das „Ländliche” als solches für das Leben in einer Stadt als essentiellen Bestandteil ansehen. Als Beispiel lässt sich die Vision des britischen Stadtplaners Ebenezer Howard mit seinem Konzept der Gartenstadt betrachten. Dieses bildet noch heute ein Vorbild für moderne Stadtplaner welches bereits 1898 erdacht wurde vertritt er das Konzept eines Zusammenspiels von ruraler Landwirtschaft und urbanen Gebieten. Er betont zudem dass es sich bei den Gärten nicht nur um angelegte Ziergärten und Erholungsorte sowie Freizeitparks handelt sondern auch um „Freiräume“, die Platz für den Anbau von Nutzpflanzen und die Aufzucht von Nutztieren bieten sollen. Neben den frühen Gedanken an eine Landwirtschaft innerhalb städtischer Strukturen ist der Aspekt der Versorgung und Infrastruktur ebenfalls betrachtet worden.\footcite[Vgl.][S. 144f]{MullerUrbanStadt}[sowie][S. 19ff]{Lohrberg2001StadtnaheFreiraumplanung}

\begin{figure}[htbp]
\centering
\includegraphics[width=10cm]{image_folder/GardenCityConcept_EbenezerHoward.jpg}
\caption{Konzept der Gartenstadt von Ebenezer Howard}
\label{fig:GardenCityConcept_EbenezerHoward}
\end{figure}

Interessant sind hier die „landwirtschaftlichen Gürtel” (ringförmige Anbauflächen um die Städte) und das Marktzentrum - den sogenanten „Crystal Palace“ im Stadtzentrum. Dies ist ein ringförmiger Markt, der einem gläsernen Gewächshaus ähneln soll, das Waren aus der umliegenden Landwirtschaft direkt an den Verbraucher vertreiben sollte. Generell bleibt festzuhalten, dass Howard bereits 1898 nicht nur ein Konzept zur Verankerung von Landwirtschaft innerhalb von urbanen Gebieten erdachte sondern auch Ideen entwickelte, wie die urbane Bevölkerung mit den Lebensmitteln versorgt werden konnte. \footcite[Vgl.][S. 19ff]{Lohrberg2001StadtnaheFreiraumplanung}

Der urban landwirtschaftliche Aspekt wurde im 20. Jahrhundert durch nachfolgende Stadtplaner oft übergangen. Für den folgenden Stadtwachstum verkam die Urbane Landwirtschaft zur Ausgestaltung von mühevoll angelegten und vermeintlich natürlichen Ziergärten, die Grünflächen und Erholungsgebiete in die Städte bringen sollten. So bleibt die Frage, wie es zu dieser gegensätzlichen Entwicklung kam.\footcite[Vgl.][S. 144f]{MullerUrbanStadt}

\subsection{Entwicklung der urbanen Landwirtschaft am Beispiel Deutschland}

Der folgende Text spiegelt einen Auszug der Entwicklung von urbaner Landwirtschaft innerhalb Deutschlands wieder. Hierbei handelt es sich um eine Beschreibung wichtiger Faktoren auf der Grundlage von Schäfers \textit{Sozialgeschichte der Soziologie.}\\
\\
In Deutschland zeigt die Sozialgeschichte das im Jahr 1800 Großteile der Bevölkerung auf dem Land leben und aus der sogenannten Agrar-Gesellschaft bestehen. Zu diesem Zeitpunkt gab es in Deutschland lediglich drei Großstädte. 1835 Endstand dann allmählich das Eisenbahnnetz welches maßgeblich zur Ausbreitung der Industrialisierung beitrug. Durch das Eisenbahnnetz konnten Arbeiter zu Ihren Arbeitsstätten pendeln, welche sich wiederum durch das wachsende Netz stärker ausbreiten konnten. Zudem konnten Güter verstärkt kostengünstig transportiert werden. Dies führte in den folgenden Jahren zum Wachstum der Städte und Gemeinden und trug zudem zum Bevölkerungswachstum bei. 1850 bildete sich die industrielle Gesellschaft heraus. Zu diesem Zeitpunkt umfasste Deutschland bereits 35 Millionen Einwohner und zählte damit als eines der größten Länder Europas. Das enorme Städtewachstum wiederum lässt sich auf die Binnenwanderungen, den Wachstum der Bevölkerung, Gebietserweiterungen und Eingemeindungen zurückführen. 1900 besaß Deutschland bereits 33 Städte mit mehr als 100 Tsd. Einwohnern. Der Ausbau des Schienennetzes ist dabei nicht der einzige Faktor oder gar der Grund der Industrialierung oder des Bevölkerungswachstums, dennoch trug es massiv zur Ausbreitung der Bevölkerung bei. Heinrich Heine, einer der bedeutendsten deutschen Schriftsteller schrieb zu dieser Zeit:\footcite[Vgl.][S. 17-21, 55-57]{Schafers2016SozialgeschichteSoziologie}
\begin{displayquote}
„Welche Veränderungen müssen jetzt eintreten in unserer Anschauungsweise und in unsern Vorstellungen! Sogar die Elementarbegriffe von Zeit und Raum sind schwankend geworden. Durch die Eisenbahnen wird der Raum getötet, und es bleibt nur noch die Zeit übrig“. (Heine, in: Lutetia)
\end{displayquote}
Dabei bot die Eisenbahn natürlich nicht nur Vorteile. Sie trug maßgeblich zur Veränderung des Landschaftsbildes bei und brachte Rus und Lärm in die Landschaften und Städte. Durch wachsende Infrastrukturen und der Industrie verdichteten sich im laufe der 20. Jahrhunderts die urbanen Gebiete massiv.\footcite[Vgl.][S. 57]{Schafers2016SozialgeschichteSoziologie}\\
\\
Lieberecht Migge schrieb 1926 in Hinblick auf diese Entwicklung „Das grüne Manifest“ in dem er das „Leiden“ (z.B hygienische Missstände und Platzmangel) der Städte zum Teil auf die Industrialisierung zurück führt. Er bezeichnet das „Land“ (hier der ländlich rurale Freiraum) als Frischluftbehälter und als universale Erweiterungszone. Interessant ist hier auch dass er das Land nicht nur als physischen Freiraum sondern auch als Freiraum für den Geist des Menschen betrachtet in dem er seine Identität frei entfalten und entwickeln kann. Leitsätze wie „Schafft Stadtland“ oder „Die Städte sollten Ihr eigenes Land umarmen“ implizieren die Grundidee des ruralen Raumes innerhalb der Stadtgebiete. „Man Pflanze!“ heißt es weiter. Aber das Gepflanzte soll auch Mehrwert erfüllen, keine einfachen nach Blumenbeet anmutenden Grünflächen um der Optik willen, sondern Parkanlagen, Spielplätze und Bäder, die der Allgemeinheit einen Mehrwert bieten. Aber auch Gedanken zur Selbstversorgung kommen auf. So schreibt er Beispielsweise von Nutzgärten mit einer größe von 80qm pro Einwohner auf denen Nahrungsmittel für den Eigenbedarf angebaut werden sollten.\footcite[Vgl.][S. 7-15]{Leberecht1926DeutscheBinnenkolonisation} Was sich zunächst nach einer Wunschvorstellung anhört, könnte in Ausnahmesituationen Leben retten.\\
\\
Nach dem zweiten Weltkrieg litten beispielsweise viele Stadtbewohner an Nahrungsmangel. In der Nachkriegszeit änderte sich daher der Blick auf Urbane Landwirtschaft massiv. In städtischen Hinterhöfen wurden zum Beispiel auch Kartoffeln und Rüben angebaut. Erholungsparks und Spielplätze wurden in Nutzgärten gewandelt da reine Grünplätze mehr und mehr als unwirtschaftlich betrachtet wurden. \\
\\
Migge war sich dessen bewusst und versuchte in seinem Manifesto eine Art Richtlinie für den Städtebau zu entwickelten. So verurteilt er auch den Wohnraum einzelner Menschen über deren Bedarf hinweg und fordert ein verstärtkes Bewusstsein für Müll und Nahrungsmittelverschwendung. Die Städter sollten in einer Art Gleichgewicht mit dem Land leben. Dies betont er wei folgt: „Die Stadt muss auch geben dem Land - will sie leben vom Land“.\footcite[S. 10]{Leberecht1926DeutscheBinnenkolonisation} Eine der wichtigsten Architektonischen Eigenschaften von Städten, das Bauen von Wohneinheiten übereinander, verurteilt er und bezeichnet das „Übereinander“ als die „Wurzel allen übels“.\footcite[S. 13]{Leberecht1926DeutscheBinnenkolonisation} ]{Leberecht1926DeutscheBinnenkolonisation} Dies ist zwar eine äußerst radikale Ansicht, repräsentiert aber den Wunsch nach freier Entfaltung und individualität. Er wünschte sich weiter ein neues Dasein im ländlich urbanen Raum welches durch harte Arbeit, Bescheidenheit und Naturverbundenheit geprägt sein sollte.\\
\\
\subsection{Urbane Landwirtschaft aus globaler Perspektive}

Betrachtet man \acs{ul} aus globaler Perspektive, so kann man ihre lange Tradition innerhalb von gesellschaftlichen und kulturellen Praktiken in Bezug auf Stadt und Gemeinschaft erkennen. Die Anfänge der \acs{ul} reichen Jahrzehnte oder Jahrhunderte zurück bis sie schließlich in ihre jetzige Form mündet. Beispiele hierfür wären unter Anderem die europäischen Schrebergärten, Gemüsefelder in afrikanischen Kolonialstädten mit Wurzeln in alten kommunalen Praktiken, das chinesische System der Weiderverwendung des Nachtbodens oder die Chinampas von Mexiko-Stadt welche ein Anbausystem vor der Ankunft von Kolumbus darstellten.

Jac Smit, Joe Nasr und Annu Ratt beschreiben in Ihrem Beitrag \textbf{Urban Agriculture Yesterday and Today} zum Werk \textbf{Urban Agriculture: Food Jobs and Sustainable Cities} von 2001 die Anfänge der Urbanen Landwirtschaft. Dort lassen sich auch einige Beispiele für globale \acs{ul} finden, welche vom Urban Agriculture Network aus verschiedenen Quellen Zusammengetragen wurden.

\begin{displayquote}
\textbf{Afrika:}
\begin{itemize}
    \item Mali: Bamako ist mit Gartenbauprodukten autark und einige Produkte werden außerhalb der Metropolregion zum Verzehr versandt
    \item Uganda: In Kampala werden 70 \% des Geflügelbedarfs (Fleisch und Eier) innerhalb der Stadt produziert.
    \item Sambia: In Lusaka macht die Nahrungsmittelproduktion 33 \% des Gesamtverbrauchs der Hausbesetzer aus.
\end{itemize}
\textbf{Asien}
\begin{itemize}
    \item China: In den 1980er Jahren wurden über 90 \% des Gemüsebedarfs und über die Hälfte des Fleisch- und Geflügelbedarfs in den 18 größten Städten Chinas durch Produkte aus städtischen Provinzen gedeckt.
    \item Indonesia: In Jakarta werden fast 20 \% der von Hausbesetzern verzehrten Lebensmittel selbst produziert.
    \item Nepal: In Kathmandu deckten 37 \% der befragten Lebensmittelhersteller ihren Bedarf an pflanzlicher Haushaltsnahrung und 11 \% den Bedarf an Tiernahrung.
    \item Singapur: Achtzig \% des Geflügels und 25 \% des verzehrten Gemüses werden in der Stadt produziert.
\end{itemize}
\textbf{Europa}
\begin{itemize}
    \item Rumänien: Mit neuen Regierungsrichtlinien und -programmen von 1992 bis 1998 stieg die städtische Produktion von 14 auf 26 \% der gesamten landwirtschaftlichen Produktion.
\end{itemize}
\textbf{Amerika}
\begin{itemize}
    \item Kuba: Von 1992 bis 2000 stieg die städtische Nahrungsmittelproduktion um 300 \% und die Kinder essen viermal so viel Gemüse wie vor einem Jahrzehnt.
    \item USA: Dreißig \% der landwirtschaftlichen Produkte des Landes werden in Ballungsräumen produziert.
\end{itemize}
\end{displayquote}

Die Anforderungen an die Produktion von Nahrungsmitteln in der Stadt sind hoch. Pflanzen und Tiere müssen strapazierfähig gegenüber den städtischen Bedingungen sein und die begrenzte und kostbare Bodenfläche erfordert den Anbau von lukrativen und hochwertigen Produkten.\\ 
\\
Im Laufe der Zeit entwickelten einige Gesellschaften Techniken und Systeme, die die Landwirtschaft als Tätigkeit in die Stadt integrierten. Andere wiederum entwickelten die Städte getrennt von der Landwirtschaft. Dieser unterschiedliche Ansatz spiegelt auch unterschiedliche Einstellungen zur Art wieder, wie natürliche und menschengemachte Umgebungen miteinander umgehen und hat kulturelle Wurzeln.\\
\\
Im späten 19. Jahrhundert begannen Maschinen die landwirtschaftliche Arbeit zu erleichtern und die Produktion sowie die Vermarktung der Erzeugnisse stieg an. Die \acs{ul} reagierte wiederum auf diese Entwicklung und spezialisierte sich beispielsweise auf Nischenmärkte, den Tauschhandel und den Devisenhandel sowie die Abfallverwertung. Die Informationsrevolution verbreitete dann das Wissen der städtischen Lebensmittelproduktion über nationale und kulturelle Grenzen hinweg und ermöglichte auch neue Formen des Marketings. Nach dem zweiten Weltkrieg schritt die Urbanisierung in manchen Ländern schneller voran als der Bevölkerungswachstum, die Wirtschaft oder Infrastrukturen. Daher fiel die Last der Lebensmittelversorgung mancherorts auf die Stadtbewohner selbst. Auf allen Kontinenten aber führte die Not an Bebauungsflächen oder Nahrungsmitteln zu intensivierter \acs{ul} und zu einem erhöhten Bewusstsein für Landwirtschaft in Hinblick auf die Ernährungssicherheit. Ob \acs{ul} nun von den ersten städtischen Siedlern entwickelt wurde um die Ernährung zu sichern oder ob sie durch die Schrittweise Veränderung der Nahrungsmittelproduktion entstand, eine Rolle spielte \acs{ul} zu jeder Zeit. Teilweise war diese sogar unerlässlich.  \footcite[Vgl.][S. 1-4]{Smit2001UrbanCities}\\
\\
\subsection{Beispiele aus der Geschichte der urbanen Landwirtschaft}

Laut Smit, Nasr und Ratt gäbe es sichere Hinweise auf groß angelegteBaumkulturen in Mayastädten. Hinweise lassen sich in Städten wie Caracol Lamanai und Belize finden. Die bevölkerungsdichte in Caracol betrug demnach ca. 1.000 Menschen pro Quadratkilometer bei einer Gesamtbevölkerung von 115.000 - 150.000 Menschen.\footcite[Vgl.][S. 5-6]{Smit2001UrbanCities}Zum Vergleich: Das \textbf{Amt für Statistik Berlin-Brandenburg} gibt an, dass am 31.12.2017 in Berlin ca. 4.055 Menschen pro Quadratkilometer in der deutschen Hauptstadt lebten.\footcite[Vgl.]{AmtfurStatistikBerlin-Brandenburg2017NoTitle}Die Stadtlandschaft von Caracol bestand aus dichten Gebäudegruppen die mit landwirtschaftlichen Terassen durchzogen waren. Forscher glauben, dass diese Archtiektur seinen Fokus auf die Selbstversorgung setzte. Die Terassen und Stauseen durchzogen die gesamte Stadt.\footcite[Vgl.][S. 5-6]{Smit2001UrbanCities}

Landwirte reaktiverten mit Hilfe von Archiologen an mehreren Standorten in Peru, Guatemala und Mexiko dei alten Systeme der \acs{ul} und erhielten interssante Erkenntnise über die damalige Landwirtschaft.\footcite[Vgl.][S. 6]{Smit2001UrbanCities}{zietiert nach Cardich und L. Fowler, 1987 National Geographic Research, Vol. 3, No.1.}
In einem Fall unterstützen die Produktionsniveaus zwei Familien auf 1 Hektar (0,40 Hektar). In einem zweiten Fall wäre eine Wiederbelebung der alten Terrassentechnik speziell erfolgreich. Eine der wichtigsten Erkenntnisse ist der Einsatz von Aqua-Terra-Systemen, in denen Wasser- und Landpflanzen in Symbiose produziert werden. Solche Systeme sind besonders relevant für die städtische Landwirtschaft, da sie in Gebieten mit weniger gutem Boden, steilen Hängen und Feuchtgebieten effektiv sind. 

\begin{figure}[htbp]
\centering
\includegraphics[width=10cm]{image_folder/SchaubildTerassenanbau.png}
\caption{Vereinfachtes Konzept des Terrassenanbaus, Eigene Zeichnung}
\label{fig:Terassenlandwirtschaft}
\end{figure}

In vielen alten Systemen würde zudem das Klima durch Techniken wie Bewässerung und Erwärmung des Bodens und der Luft gemildert, um die Vegetationszeiträume zu verlängern. In Europa würde beispielsweise Kompost, einschließlich Pferdemist, schon seit langem zur Beheizung von erhöhten Gemüsebeeten verwendet. Seit hunderten Jahren und in verschiedenen Teilen der Welt wären Anbau und Viehzucht innerhalb und außerhalb der Stadtmauern Standard. Bevor in der zweiten Hälfte des 19. Jahrhunderts moderne städtische Abwassersysteme entwickelt wurden, wäre die städtische Landwirtschaft die wichtigste Behandlungs- und Entsorgungsmethode für städtische Abfälle gewesen.\footcite[Vgl.][S. 6-7]{Smit2001UrbanCities}

Die Liste der Arten von urbaner Landwirtschaft in alten Kulturen ist lang und beschreibt die Wichtigkeit von marktnaher Landwirtschaft vor der Erfindung von Kühlsystemen oder schnellen Transportmitteln. Aber auch in späteren Jahren beweist die urbane Landwirtschaft ihre Relevanz.

Ernst May erdachte beispielsweise 1922/23 Trabantenstädte erdachte zu, die von „Kulturbändern“ umgeben wurden, auf diesen dann eine intensive Landwirtschaft entstehen sollte. Gemüse und Kleinvieh von Kleinbauern sollte so herangezogen und gezüchtet werden, welches dann als Nahrungsmittel für die Siedlungen gefördert werden sollte. Dieser Ansatz floss teilweise in den 1920er Jahren in den Städtebau mit ein und stärkte dabei das Kleingartenwesen. \footcite[Vgl.][S. 145]{MullerUrbanStadt}\\
\\
\begin{displayquote}
„Häufig werden zwei „Ursprungsorte“ des Urban Gardening genannt: Zum einen Kuba, das nach dem Lieferstopp von günstigem Erdöl aus der Sowjetunion 1989 die Landwirtschaft auf postfossile Bewirtschaftung umstellen musste. Dabei kam der urbanen Landwirtschaft eine zentrale Rolle für die Überlebensproduktion zu. Der andere häufig genannte Ort ist das New York der 1970er Jahre, wo AktivistInnen mit Guerilla Gardens und Community Gardens die Lebensbedingungen in vernachlässigten Stadtvierteln verbessern wollten. Sie gelten als eine frühe Form der urbanen Intervention und des politischen Protests. Als Genese der neuen urbanen Gärten in Deutschland können die Community Gardens jedoch nicht angesehen werden, hier gab es seit Mitte der 1990er Jahre eine eigenständige Entwicklung durch die Interkulturellen Gärten, die aus der Migrationsbevölkerung selbst entstanden.“ \textit{Dr. Christa Müller}\footcite{}
\end{displayquote}\\
\\
Zusammenfassend lässt sich sagen, dass urbane Landwirtschaft häufig als Versorgungsrettung betrachtet wird. Das Bewusstsein für die vermeintlich lebensrettende Methode der Landwirtschaft scheint immer dann stark zu wachsen wenn es den Bewohnern urbaner Gebiete besonders schlecht ergeht. Dies wurde auch 2007 deutlich als schwere Waldbrände in Griechenland wüteten. Die Krise führte zu einem vermehrten Anbau von Nahrungsmitteln innerhalb der griechischen Städte und führte sogar zur Ausbeutung von städtisch landwirtschaftlichen Räumen durch Zivilisten. Seit 2012 wird die urbane Landwirtschaft von der Zivilgesellschaft oder den einzelnen Gemeinden verbreitet.\footcite[Vgl.]{Kolokouris2015URBANPROTEST}[sowie]{Bormann2007FeuerOlympia}

Ein weiteres Beispiel für den Antrieb von Urbaner Landwirtschaft in Kriesenzeiten lieferte Kuba in den 90er Jahren. Durch den Zusammenbruch der Sowjetunion kam es auf der Insel auch zum Kollaps der dortigen Wirtschaft, da diese stark von der sowjetischen Ölversorgung abhing. Durch den fehlenden Rohstoff konnten beispielsweise keine Erntemaschinen betankt werden. Zudem mangelte es auch an Düngern und Pestiziden. Die Einwohner der Insel setzten immer stärker auf die urbanen Anbaumethoden. Heute gehört der Inselstaat zu den Ländern die große Fortschritte zur Bekämpfung von Hunger durch Urbane Landwirtschaft gemacht haben. \footcite[Vgl.]{Sieg2016DieKuba} Die Stadt Detroit geriet nach Jahrzehnten einer starken Wirtschaft in eine schwere Krise. Die einstige florierende Automobilbranche liegt brach. Dadurch zogen viele arbeitslose Einwohner aus der Stadt. Durch die ungenutzten urbanen Gebiete konnte aber in den darauf folgenden Jahren eine aufsteigende Landwirtschaft entstehen. Brachland und Dachterrassen wurden nun zu ca. 1200 fruchtbare Gärten verwandelt.

\begin{displayquote}
„Nach einer Studie der Michigan State University könnte Detroit mit Stadtfarmen, Nachbarschaftsgärten und Gewächshäusern drei Viertel ihres Gemüses und vierzig \% ihres Obstes selbst produzieren.“\footcite{Sieg2016GemuseDetroit}
\end{displayquote}

\section{Aktuelle Bedeutung der urbanen Landwirtschaft}
\subsection{Aktuelle Aufgaben und Motivation}

Die letzte quantitive Aussage wie sehr \acs{ul} derzeit auf der Welt verbreitet ist bietet Smits Statistik aus dem Jahr 2000 (Siehe Abbildung \ref{fig:ulstatistics}):\footcite[Vgl.][S.1]{Smit2001UrbanToday}
\begin{figure}[htbp]
    \centering
    \includegraphics[width=10cm]{image_folder/800000ul.png}
  \caption{Schätzungen des Urban Agriculture Network basierend auf dem Erfahrungen und Beobachtungen von Jac Smit}
  \label{fig:ulstatistics}
\end{figure} 

Aufgaben und Antriebe von \acs{ul} drehen sich noch heute um die Frage der Lebensmittelversorgung. Je nach Region ist sie von subsistenzieller oder sozialpolitischer Natur. Wie gut ein Land versorgt ist hängt von vielen Faktoren ab. Darunter zählt der wirtschaftliche Status, die Verfügbarkeit von Ressourcen, die jeweilige Ernährungspolitik und wie gut die Infrastruktur zur Lebensmittelversorgung (Transport, Lager, Vermarktung und viele mehr) ist. Demnach unterscheiden sich die Aufgaben, die urbaner Landwirtschaft zugewiesen werden je nach dem Entwicklungsgrad des umgebenden Landes.\footcite{Smit2001UrbanToday} In Industrieländern ist sie von soziokultureller und politischer Natur.\footcites[][S. 21]{Berges2014UrbaneStadt}[S.26f]{Smit2001UrbanToday} In Entwicklungsländern hingegen dient sie vorwiegend zur Subsistenz oder als zusätzliches Einkommen für die Stadtbewohner. \footcites[Vgl.][S.75]{Nugent2000TheEconomies}[S.26f]{Smit2001UrbanToday}\\
\\
Warum urbaner Gartenbau noch bedeutend für Entwicklungsländer ist, erläutert Moustier: \footcites[Vgl.][S.6ff]{Moustier2007UrbanSupplier} Gängig sei der Anbau von schnell verderblichen Lebensmitteln wie Blattgemüse oder Tomaten. Durch die Nähe zur Stadt ist der Transportweg in der Regel kürzer und die Produkte gelangen zeitnah zum Konsumenten. Zusätzlich ist in den meisten Fällen die Lieferkette kürzer, was den Preis der Lebensmittel senkt. Bei weniger verderblichen Lebensmitteln schwanke der Anbau zwischen den Stadtrandgebieten und ruralem Anbau. Den Konsumenten sei auch die Frische der Produkte besonders wichtig, da ihnen oft eine geeignete Lagerung wie Kühlschränke fehle. Bedenklich sei allerdings die Qualität der Lebensmittel durch die verunreinigte Wasser- und Bodenqualität sowohl in der Stadt als auch im Umland. Jedoch sorge die Nähe für einen besseren Informationsaustausch zwischen Konsument und Produzent zum Anbau. Das erleichtert den städtischen Gemeinden und den Menschen vor Ort die Kontrolle ihrer Ernährungssicherheit. Darüber hinaus habe die Entstehung urbaner Landwirtschaft oft damit zu tun, dass viele Städte geschichtlich an rohstoffreichen Gegenden angesiedelt waren und das Umland von Städten daher gute Anbaubedingungen vorweist. Die saisonalen Bedingungen spielen, Moustier zufolge, ebenfalls eine wichtige Rolle. In ungünstigen Jahreszeiten sind Stadtbewohner angewiesen auf die Zufuhr von ferner angebauten Nahrungsmitteln. \\
\\
Auch der private Anbau in diesen Städten sei nicht zu vernachlässigen. Zu der Frage, warum Stadtbewohner in diesen Regionen \acs{ul} betreiben, ergaben sich aus Nugents Umfrage im Jahr 2000 mehrere Antworten. \footcite[Vgl.][S.74]{Nugent2000TheEconomies} Zu den Hauptgründen zählte der Eigenverbrauch, die Einkommenssteigerung und ökonomische Krisen. Besonders die Unsicherheit der Ernährungslage sei eine häufige Sorge, die Stadtbewohner antreibe. Zum Beispiel in Accra (Ghana) sei Viehbestand eine sichere Anlage in schweren Zeiten. Auch die Höhe des Einkommens beeinflusse den Aufwand, den jene Haushalte für \acs{ul} betreiben. Zezza et. al. zeigten durch ihre Datenerhebung wie aktuell\footnote{Diese Datenerhebung fand 2010, also zehn Jahre nach Nugents Untersuchungen statt. Der Grund, den Zezza et. al. für ihre Arbeit nannten, war der Mangel an repräsentativen Daten dieses Forschungsbereichs.} Nugents Punkte noch sind:
\begin{displayquote}
„With very few exceptions, a clear negative correlation between participation in agricultural activities and level of welfare is noted. Participation rates for the poorest quintile are extremely high, over 50\% in 8 out of 15 countries, proving how urban agriculture plays an important role for a non-negligible number of poor households in the developing world.“\footcite[S.268]{Zezza2010UrbanCountries}
\end{displayquote}

Diverse Organisationen setzen sich für den Einsatz von \acs{ul} als Lösung für Unterernährung und Armut in Entwicklungsländer ein, eines der größten ist wohl die Ernährungs- und Landwirtschaftsorganisation der Vereinten Nationen (\acs{fao}).\footcite[Vgl.][S.6ff]{FAOAboutNations} In diesem Zusammenhang äußert sich \acs{fao} Generaldirektor Jaques Diouf:

\begin{displayquote}
„\lbrack...\rbrack urban poverty tends to be fuelled by people migrating towards the cities in an attempt to escape the deprivations associated with rural livelihoods. Partly due to the rural decline, the world is urbanizing at a fast pace and it will not be long before a greater part of developing country populations is living in large cities. Therefore, urban food security and its related problems should also be placed high on the agenda in the years to come.“\footcite[S.5f]{FoodandAgricultureOrganizationoftheUnitedNations2006The2006}
\end{displayquote}

\begin{figure}[htbp]
    \centering
    \includegraphics[width=14cm]{image_folder/simplified_hydroponics_jack_sanford.jpg}
  \caption{„Simplified Hydroponics“ von Jack Sanford}
  \label{fig:shydroponics}
\end{figure} 

\subsubsection*{Beispiel: Vereinfachte Hydrokultur in Ecuador}
Besonders seit 1991 fördere die \acs{fao} Workshops zum vereinfachten Hydrokulturanbau (Pflanzenanbau mit Substraten anstelle von Erde).\footcites[Vgl.][o.P. (S.1)]{Stajano2003SIMPLIFIEDEcuador} Die  Hydrokultur unterscheide sich von hochwertigen Variante in dem Sinne, dass sie - im Vergleich zu industriellen Produktionen - ausgelegt ist auf die Ernährung von gering verdienenden Familien und daher wenig Ressourcen verwendet. Vorteile dieser Methode seien unter anderem die einfache Erlernung, die Vermeidung von verseuchtem Boden und flexible Anbaustellen in Städten. Desweiteren können alte Behälter dafür wiederverwendet werden und der Anbau erfordere weitaus weniger Wasser als üblicher Feldanbau. Ein Beispiel dazu ist das im Jahr 2000 umgesetzte Hydrokulturprojekt in den urbanen Regionen Ecuadors.\footcite[Vgl.][o.P. (S.2f)]{Stajano2003SIMPLIFIEDEcuador}. Ziel des Projektes war es Kindern unter sechs Jahren Zugang zu qualitativen Lebensmitteln zu bieten und einkommensarmen Familien durch diese Seminare nicht nur diese Anbaumethode beizubringen sondern auch einen besseren Lebensstandard zu bieten und im Idealfall ein sicheres Einkommen zu ermöglichen. Schlussendlich wurden nicht nur die aufgelisteten Ziele erreicht, darüber hinaus wurden die betroffenen Kinder durch die gesunde Ernährung seltener krank. Das Projekt kam so gut bei den jeweiligen Gemeinden an, dass Interesse bestand den erlernten Anbau eigenständig auszubauen. \\
\\
Im vorherigen Absatz wurde die subsistenzielle Bedeutung von UL in Entwicklungsländern erläutert. In Industriestaaten sei hingegen die Unsicherheit über die Lebensmittelversorgung nicht so stark wie in Entwicklungsländern. Smit et. al. geben dafür drei Gründe an: Zum einen sei das Verhältnis zwischen den Lebensmittelkosten und dem urbanen Haushaltsbudget in Industrieländern geringer. Während Stadtbewohner im globalen Norden ein Drittel bis ein Fünftel ihres Haushaltsbudgets für Essen verbrauchen, liegt der Anteil der Lebensmittelkosten in ärmeren Ländern zwischen ein Drittel und vier Fünftel. Ein weiterer Grund sei, dass die Infrastruktur zur Nahrungsmittelversorgung vollständiger sei. Und der dritte Punkt sei die höhere Lebensmittelqualität aber auch, dass sie zugänglicher seien für die Stadtbevölkerung.\footcites[Vgl.][S.27]{Smit2001UrbanToday} Smit et. al. zufolge wachse das Interesse an urbanen Gemeinschaftsgärten in Europa und den Vereinigten Staaten. Der Antrieb dafür sei unter anderem die Sorge um die Nahrungsmittelqualität der globalen Lebensmittelkonzerne. Angeknüpft dazu seien soziale politische Motive. Ferner führte das Leibniz-Zentrum für Agrarlandschaftsforschung 2011 bis 2014 eine Innovationsanalyse zu urbaner Landwirtschaft in den Vereinigten Staaten und Deutschland aus. Eine Erkenntnis war, dass heutige \acs{ul}-Formen in den untersuchten Ländern vorwiegend soziale Neuerungen zeigen \footcite{Berges2014UrbaneStadt}. Passend dazu ist die mediale Wahrnehmung des Begriffs Urban Gardening in Deutschland, den Egnolff vor drei Jahren untersuchte. Ihren Analysen zufolge werde dem Begriff zwar auch ökonomische Bedeutung zugewiesen, allerdings seien die politische, soziale als auch nachhaltige Bedeutung besonders relevant.\footcite[Vgl.][S.119ff]{Egnolff2015DieIdeal}. Ein passendes Beispiel dazu ist das gemeinnützige Projekt The Plant in Chicago.



\begin{figure}[htbp]
    \centering
    \includegraphics[width=14cm]{image_folder/the_plant_1.jpg}
    \caption{The Plant im Zentrum Chicagos}
\end{figure} 

\begin{figure}[htbp]
    \centering
    \includegraphics[width=14cm]{image_folder/the_plant_2.jpg}
    \caption{Hydroponischer Plfanzenanbau in The Plant}
\end{figure} 

\subsubsection*{Beispiel: The Plant}
Das im Zentrum Chicagos gelegene The Plant war vor seiner aktuellen Gebäudenutzung eine Fleischverarbeitungsanlage. Dank staatlicher Unterstützung wurde aus dem 8000 m\textsuperscript{2} alten Warenhaus eine energieneutrale Einrichtung für eine nachhaltig orientierte Gemeinschaft von kleinen Lebensmittelunternehmen. Dazu zählen Landwirtschaftsbetriebe, die innen und außen anbauen, eine Kafferösterei eine Bäckerei, sowie eine Brauerei. The Plant wird auch als Vertical Farm bezeichnet. Auf mehreren Etagen des Gebäudes werden unter anderem Pflanzen, Gemüse und Pilze angebaut. Ergänzend dazu werden regelmäßig Kurse zum Anbau und zur Ernährung gegeben. Aufgrund des energieneutralen Gebäudekonzeptes ordnen Berges et. al. dieses Projekt als innovativ ein. Geplant sei nämlich eine Biogasanlage, die ausreichend Lebensmittelabfall in Elektrizität und Wärme verbrennt, um das Haus ohne zusätzliche Ressourcen zu heizen und zu kühlen. Die finale Realisierung des Gebäudes erfolge 2019.\footcites[Vgl.][S.18]{Al-Kodmany2018TheCity}{TheLLC}\\
\\
Bisher wurden die subsistenziellen und soziokulturellen Motive aktueller \acs{ul} beleuchtet. Untersucht man die Motive technologischer Neuentwicklungen urbaner Landwirtschaft, ist dieser jedoch vor allem angetrieben durch zukünftige Herausforderungen.\footcites[Vgl.][S.238]{Germer2011SkyfarmingSecurity}[sowie][S.6]{PeterMollerVoss2013VerticalRise}[][S.1]{Al-Kodmany2018TheCity} Einer der Hauptgründe sei das steigende Bevölkerungswachstum und die damit zusammenhängende weltweite Verstädterung. 2050 soll die Bevölkerungszahl von jetztigen sieben auf neun Milliarden Menschen ansteigen. Davon werden 68\% der Menschen in Städten leben.\footcite[Vgl.][S.3]{UnitedNationsDepartmentofEconomicandSocialAffairs2017World2017} Das führt wiederum zum Ausbau von Siedlungs- und Verkehrsflächen, was allerdings landwirtschaftliche Flächen reduziert. Der herkömmliche Feldanbau käme dadurch an seine Grenzen und somit auch die Lebensmittelproduktion, die bei steigender Bevölkerungszahl ebenfalls steigen müsste. Es stellt sich also die Frage, wie ein erhöhter Lebensmittelbedarf trotz begrenzter Anbaufläche einerseits gedeckt werden kann, andererseits aber auch ökoeffizient ist und somit zu einer nachhaltigen Entwicklung beiträgt. 
\\
\\
In Anbetracht der zentralen Fragestellung dieser Forschungsarbeit stehen vor allem Landwirtschaftskonzepte, die auf umfangreiche Lebensmittelproduktion ausgelegt sind, im Fokus der Untersuchung. Diese werden in Kapitel 10 vorgestellt und nach den Nachhaltigkeitskriterien im Abschnitt \ref{nachhaltigkeitskriterien} in geprüft.

\subsection{Aktuelle Lebensmittelversorgung in Städten}
Laut Stierand hat die Erzeugung und Verarbeitung von Lebensmitteln in der heutigen Zeit „ihre Bindung an die Verbrauchsräume weitgehend verloren, schnelle und gute Transportmöglichkeiten erlauben weltweiten Handel. So ist die Landwirtschaft in und um Städte nicht mehr überlebenswichtig“.\footcite[S.122f]{Stierand2008StadtLebensmittel} 

\begin{figure}[htbp]
\centering
\includegraphics[width=12cm]{image_folder/ernahrung.png}
\caption{Änderung des Ernährungsmaßstabs}
\label{fig:massstab}
\end{figure}

Mit der Entstehung von regionalen und nationalen Ketten dehnte sich der Handel aus und das Ernährungssystem erfuhr eine „Delokalisation“  \footcite{MASSIMOMONTANARI1993DerEuropa}[188ff], wie bereits in Kapitel \ref{entwicklungglobal} beschrieben wurde. Diese Veränderung durch die Ausbreitung der Filialen und der folglich flächendeckenden Versorgung wirkte sich sowohl in städtischen als auch in ländlichen Regionen aus. Seitdem ist „die Bindung zwischen Wohnort und Nahrung [...]aufgehoben, die Ernährung ist nicht mehr von saisonalen Engpässen und Überschüssen abhängig.“\footcite[S.122]{Stierand2008StadtLebensmittel}  

\subsubsection {Ernährung in der Stadt} \label{StadtLebensmittel}
Heute erreichen Lebensmittel den Konsumenten zu 90\% in verarbeiteter Form. Vor allem der Anteil von „Convenience Produkten, also Fertig- oder Halbfertigprodukten habe laut Hemmerling stark zugenommen.“ (Hemmerling et al., 2012, p. 33). „Die zeitaufwendigen Prozesse der Verarbeitung und Zubereitung wurden zunehmend in die Industrie transferiert.“  
Die dauerhafte und bewusste Essensplanung bezüglich saisonaler Verfügbarkeit von Lebensmitteln wurde durch den globalen Handel und die folglich beinahe ständige Verfügbarkeit der Lebensmittel überflüssig.\footcite[Vgl.][S.20]{SchmidtDieVon}
Laut Stierand wurde der Mensch erst durch das moderne Ernährungssystem zum ausschließlichen Verbraucher von Lebensmitteln. Er begründet dies mit der Aussage: „Wir sind Konsumenten. Wir essen nicht mehr selbstgezogene Lebensmittel, sondern wir verzehren Produkte, deren Herkunft, deren Produktion und Geschichte uns nur in Ausnahmefällen bekannt ist. Essen als Resultat einer arbeitsteiligen Nahrungskette […] wird uns fremd.“\footcite{Spiekermann2000GesundeKulturwissenschaft} 
„Die Mehrzahl der heutigen Konsumenten habe weder direkten Kontakt zu den ursprünglichen Erzeugern, noch haben sie einen Einblick in die Prozesse der Verarbeitung und Zubereitung.
Die Ernährung beginnt für die meisten im Warenregal von Supermärkten, Fachgeschäften oder Discountern.“\footcites[S.20]{SchmidtDieVon}[Vgl.]{BerichtInhalt}\\
\\
 Auf eine detailierte Beschreibung der Essgewohnheiten der städtischen Bewohner in der heutigen Zeit kann in diesem Rahmen nicht eingegangen werden. Im folgenden wird daher versucht die Hauptentwicklungen bezüglich der Nachfrage an Lebensmittel am Beispiel Deutschland aufzulisten.\\
 \\
 \begin{itemize}
\item Convenience-Produkte: „Fertig- und Halbfertiggerichte haben sich bereits so fest in unserer Ernährung etabliert, dass sie nicht mehr als solche wahrgenommen werden: Nudeln und Joghurt kommen zu fast 100\% aus industrieller Produktion, auch Brot wird in fast keinem Haushalt mehr selbst hergestellt. [...] Bestimmte Fertigprodukte (wie zum Beispiel der Hamburger)
haben eine eigene Identität bekommen, bei ihnen denkt niemand mehr an die eigentlichen Rohprodukte.“ \footcite{Escher2003EssenKultur}
\item Funktionelle Lebensmittel: Darunter werden Produkte definiert, die über ihre Nährstoffe hinaus eine gesundheitsfördernde Wirkung haben.\footcites[Vgl.][S.5]{Heasman1958-2001TheProfits} „Prominentes, aktuelles Beispiel sind probiotische Joghurts, die durch spezielle Joghurtkulturen eine Steigerung der Abwehrkräfte versprechen.“\footcite[S.14]{Stierand2008StadtLebensmittel}
\item Produkte aus ökologischem Anbau: In Deutschland leben „40\% der Bio-Käufer [...] in Großstädten, 27\% in mittleren und 33\% in kleineren Städten beziehungsweise Orten.“\footcite[S.14]{Stierand2008StadtLebensmittel}
\end{itemize}
Schmidt fasst die Veränderungen in der Lebensmittelversorgung wie folgt zusammen. Da sich, wie bereits genannt, seit der Entstehung der Filialen die ländliche und städtische Versorgung einander angeglichen haben, beziehen sich die folgenden Punkte generell auf städtische und ländliche Bewohner in Industrieländern.

\begin{itemize}
\item Die „Ernährung kann gesichert werden ohne selbst als Produzent tätig werden zu
müssen.“\footcite[S.20]{SchmidtDieVon}
\item Arbeitsverhältnisse sind unabhängig von der häuslichen Versorgung.
\item Saisonale oder regionale Verfügbarkeiten werden durch den globalen Handel
ausgeglichen, so dass nahezu jederzeit alle Lebensmittel zur Verfügung stehen.
\item Lebensmittel verarbeiten und zubereiten zu können ist keine Notwendigkeit für
eine (warme) Mahlzeit.
\item „Die städtische Ernährung und die ländliche Ernährung unterscheiden sich nicht“\footcite[S.20]{SchmidtDieVon}
\end{itemize}

\subsubsection{Die Rolle des Konsumenten}
Seit der allgegenwärtigen Verfügbarkeit von Lebensmitteln bleibt dem Konsument hinsichtlich seiner Ernährung lediglich die Aufgabe sich zwischen der Vielzahl von Produkten zu entscheiden. Gemäß Schmidt appelliert der Handel ab diesem Punkt an die Ernährungskompetenz des informierten Bürger.\footcite[S.20]{SchmidtDieVon} „Seit der Proklammierung der Agenda 21 (UN, 1992) wird erwartet, dass der Verbraucher die Nachhaltigkeit seines Ernährungsverhaltens bedenkt.“\footcite[S.20]{SchmidtDieVon} Gemäß Reisch ist nachhaltiger Konsum „die Abwägung vieler Faktoren“. 
Sustainable food consumption is a choice for food, which is beneficial and life enhancing for individuals, society and the planet. \footcite{article}
Die Verantwortung, die der Konsument schlussendlich übernimmt ist laut Schmidt sowohl persönlich als auch global.\footcite{SchmidtDieVon}

Zwar besteht die Bereitschaft vieler Menschen mehr Geld für eine nachhaltige Ernährung auszugeben,\footcite[Vgl.][S.10f]{Rodiger2015HowReview} gleichzeitig überschätzen sich die Konsumenten diesbezüglich, da sie überhaupt nicht mehr fähig sind zwischen den Risiken und Auswirkungen auf ihre Ernährung sowie die ökologische Entwicklung der Umwelt zu unterscheiden und diese zu bewerten. \footcites[Vgl.][S.10f]{Rodiger2015HowReview}[S.121f]{Stierand2008StadtLebensmittel}\\

\subsection{Flächenbedarf von Ernährungsgewohnheiten} 

Die Ernährung der Weltbevölkerung ist stark vom Gebrauch der weltweiten Ackerfläche abhängig. Hinzu kommt dass Teile der Weltbevölkerung andere Ernährungsgewohnheiten verfolgen und es somit zu einem unterschiedlich stark ausgeprägten Konsum von Nahrungsmitteln kommt. Wie in Abbildung \ref{fig:lFlaeche} zu sehen ist, stehen weltweit etwa 38 \% der vorhandenen Landfläche (d.h. ungefähr fünf Millarden ha) für landwirtschaftliche Nutzung zur Verfügung. Der größte Teil mit 69\% hiervon ist Weideland (mit ungefähr 3,4 Mrd. ha). 28\% der landwirtschaftlichen Nutzfläche ist Ackerland und nur 3\% Dauerkulturen.\footnote{Unter Dauerkulturen werdenKulturen bezeichnet, die mindestens für fünf Jahre auf der Fläche bestehen. Beispiele sind Obst- und Gemüsearten, wie Äpfel, Birnen, Kirschen oder auch Nüsse}
Da etwa ein Drittel des Ackerlands der Futtermittelproduktion dient, entsteht so ein Wert von 80\% der gesamten landwirtschaftlichen Nutzfläche, der für die Tierhaltung gebraucht wird. Von diesen 80\% der landwirtschaftlichen Fläche wird allerdings nur ein Anteil von 17\% 
%(3,4 Mrd. ha)
der weltweiten Nahrungsmittelversorgung gedeckt.\footcites[]{2008FAOSTAT}[Vgl.][S.6]{VonKoerber2008Globale-trends}

%von Spebs Anfang
%Laut \acs{fao} betrug im Jahr 2017 die daraus entstehende weltweite Fleischproduktion ungefähr 330.4 Millionen Tonnen. Die \acs{fao} prognostiziert bis Ende 2018 einen Anstieg auf 336.2 Millionen Tonnen.\footcite[Vgl.][S. 7]{Bedford2017FoodMarkets} Das statistische Bundesamt veröffentlichte 2010 einen Bericht zur Flächenbelegung von Ernährungsgütern in dem die Flächennutzung für Nutzvieh weltweit auf ca. 13.123.000 Hektar Land geschätzt wird. Das entspricht einer Fläche von ca. 131.230.000.000 m\textsuperscript{2}n. Der Futterverbrauch der Tiere lag dabei bei ca. 130.234.000 Tonnen.\footcite[Vgl.][S. 9]{StatistischesBundesamt2010Flachenbelegung2010} Die Menge der Fleischproduktion betrug in diesem Jahr ca. 293 Millionen Tonnen.\footcite[Vgl.][S. 9]{Sharma2014TheImpacts} 
%von Spens Ende

%Mein Teil 
%Der weltweite Konsum von Fleisch, Milch, Milchprodukten und Eiern soll laut Prognosen weiter steigen. \footcite[Vgl.][S.4f]{VonKoerber2008Globale-trends} Der Fleischkonsum soll sich bis zum Jahr 2050 weltweit verdoppeln und der Milchkonsum um ein Viertel ansteigen. Hierbei bestehen jedoch große regionale Unterschiede.\footcite[Vgl.][S.4]{VonKoerber2008Globale-trends} Während sich der Verbrauch in den Industrieländern bereits einer Sätttigung annähert, steigt der Konsum in Schwellen und Entwicklungsländern noch lange. Der Grund dafür ist dort der Anstieg der Lebensqualität und ein daraus folgender steigender Konsum von Luxusprodukten.
%

\\Getreide habe laut Von Koerber einen Anteil von 50\% des Nahrungsmittelkonsums weltweit. Das lasse sich vor allem darauf zurückführen, dass in Entwicklungsländern dieser Anteil bis zu 80\% ausmache.\footcite[Vgl.][S.4ff]{VonKoerber2008Globale-trends}
Betrachtet man die Entwicklung aller „Verwendungen (einschließlich Nahrung, Futtermittel und andere Verwendungen, wie Saatgut oder Produktion von Ethanol und Stärke), soll[e] der weltweite Verbrauch von Getreide ansteigen, von 309 kg pro Person im Jahr 2000 auf zu erwartende 339 kg bis 2050.“\footcites[S.3ff]{VonKoerber2008Globale-trends}[vgl.][S.23ff]{FAO2006World2030/2050} 
Laut Keyzer et al. zufolge würden die Prognosen „die Nachfrage nach Getreide als Futtermittel oft stark unterschätzen. Demnach könnte die Futtergetreidenachfrage in den kommenden Jahrzehnten höher sein, als bis jetzt prognostiziert.”\footcite[S.3f]{VonKoerber2008Globale-trends}Andere Quellen relativieren diese Aussage und sprechen von einer früheren Sättigung als erwartet.\\

\begin{figure}[htbp]
\centering
\includegraphics[width=16cm]{image_folder/2030S8.png}
\caption{Weltweiter Anstieg des Lebensmittelkonsums bis 2050}
\label{fig:konsumbis2050}\footcite[vgl.][S.8]{FAO2006World2030/2050} 
\end{figure}

\begin{figure}[htbp]
\centering
\includegraphics[width=16cm]{image_folder/KonsumWeltweit.png}
\caption{Weltweiter Konsum von Lebensmitteln}
\label{fig:konsumweltweit}\footcite[S.4f]{VonKoerber2008Globale-trends}
\end{figure}


Durch den Anbau von Lebensmittel, wird der Boden entsprechend belastet. Die eingesetzten Maschinen zur Bodenbearbeitung und Ernte sowie der Einsatz von Pflanzenschutz- und Düngemitteln hinterlassen Spuren im Boden, Wasser und in der Luft. So beschleunigt besonders die auf Ertragssteigerung ausgerichtete Landwirtschaft Bodenerosion und Bodenunfruchtbarkeit.\footcite{UmweltbelastungenUmweltbundesamt}
\begin{figure}[htbp]
\centering
\includegraphics[width=10cm]{image_folder/LFlaeche.png}
\caption{Weltweite Nutzung von landwirtschaftlicher Fläche}
\label{fig:lFlaeche}\footcite[Eigene Darstellung in Anlehnung an]{2008FAOSTAT}
\end{figure}

Zusätzlich wird wie in Abbildung \ref{fig:AckerproKopf} zu beobachten ist, mit dem weiteren Wachstum der Weltbevölkerung die vorhandene Ackerfläche pro Kopf weiterhin sinken. 
\begin{figure}[htbp]
\centering
\includegraphics[width=10cm]{image_folder/weltbevoelkerung_anbauflaeche_pro_kopf.jpg}
\caption{Weltweite vorhandene Ackerfläche pro Kopf}
\label{fig:AckerproKopf}
\end{figure}



\section{Vertikale Landwirtschaft}

Eine Form der UL ist die Vertikale Landwirtschaft (im Folgenden VL genannt). Die VL ist ebenso bekannt unter der englischen Bezeichnung Vertical Farming. Sie beschreibt den Anbau von pflanzlichen oder tierischen Nahrungsmitteln in und an Gebäuden, wie Hochhausfassaden, -dächern oder den mehrstöckigen und damit dreidimensionalem Anbau in sogenannten vertikalen Farmen, welcher auf kommerzieller Ebene betrieben wird. Im Zusammenhang mit vertikaler Landwirtschaft wird häufig der Begriff Zfarming (Zero Acreage)\footnote{engl. für Nullfläche} verwendet, welcher darauf verweist, dass für den vertikalen Anbau keine herkömmliche landwirtschaftliche Fläche des Erdbodens benötigt werden. \footcite{StadtischeFarming} Stattdessen werden Dachgärten, Gewächshäuser auf Dächern, vertikale „Essbare Wände\footnote{die englische Bezeichnung lautet hierfür edible wall}” Indoor- und Outdoor-Gärten genutzt um Nahrungsmittel anzubauen.

Grundsätzlich wird vertikale Landwirtschaft in drei Typen unterteilt:

\begin{itemize}
    \item Privat oder sozial motivierter urbaner Gartenbau: Dieser entsteht hauptsächlich aus gemeinschaftlichen oder politischen Gründen (siehe Kapitel \ref{UG}).
    \item Gebäudeintegrierte Landwirtschaft\footnote{engl.: building-integrated agriculture}: Sie beschreibt die generelle Integration eines Anbausystems auf oder an Gebäuden. Auf diese Weise werden Synergien zwischen Gebäude und Agrikultur möglich, wie beispielsweise Energie- oder Lebensmittelerzeugung.
    \item Kommerzieller Anbau in Indoor Farmen: Mittels großflächiger Anbausysteme werden hier ganzjährig Nahrungsmittel produziert.
\end{itemize}
    
  


\subsection{Anbaumethoden}
Dass \acs{vl} grundsätzlich möglich ist, liegt an bodenunabhängigen Anbaumethoden wie Hydroponik, Aeroponik oder Aquaponik. Al-Kodmany nach versprechen sie eine große Entwicklung in der Zukunft von \acs{vl}. \footcite[S.1]{Al-Kodmany2018TheCity} Allen gemein haben diese Methoden, dass sie dem traditionellen Feldanbau gegenüber den Wasserverbrauch reduzieren und höhere Erträge gemessen am Flächenbedarf produzieren.\footcite[Vgl.][S.7f]{Al-Kodmany2018TheCity} Im Besonderen wird in diesem Abschnitt Aquaponik untersucht. Tyson zufolge nähere sich dieses Verfahren sogar sehr stark an der Definition nachhaltiger Landwirtschaft an\footcite[Vgl.][S.36]{TysonV.2007ReconcilingMedium}. Die Frage stellt sich allerdings, um was es sich bei Aquaponik handelt. Dazu ist die Definition von Hydroponik zunächst hilfreich.\\
\\
Hydroponik bezeichnet eine Methode, die anstelle von Erde ein anderes Medium, wie mit Nährstoffen angereichertes Wasser zum Pflanzenanbau verwendet. Bei Aquaponik handelt sich um ein Kreislaufsystem, das Aquakultur(Fischzucht) mit Hydroponik vereint. \footcite[Vgl.][S.44f]{Spring2012DerBasel-Stadt} Alleinstehend haben beide Systeme - Fischzucht und Hydroponik - einige Nachteile. Industrielle Hydroponik erfordere einen hohen Umfang an Chemikalien für die Nähstoffzufuhr.\footcite[Vgl.][S.8]{Al-Kodmany2018TheCity} Außerdem sei für eine Fischzucht eine regelmäßige Wasserreinigung notwendig wodurch große Wassermengen verbraucht werden. Die Verknüpfung der beiden Methoden lösen allerdings die genannten Nachteile auf: Die Fäkalien der Fische reichern das Wasser mit Nährstoffen an, was zu den Pflanzen gelangt und ihnen als organischer Dünger zugute kommt. Die Pflanzen wiederum säubern das Wasser von Gasen, Säure, Nitrate und Phosphate, wodurch das Wasser für die Fische wiederverwendet werden kann.(Siehe Abbildung \ref{fig:aquaponik}) \\
\\
Turcios et. al. nach existierte diese Anbaumethode bereits vor 8000 Jahren. Erste Aquaponiksysteme seien im Süden Chinas und Thailand zu finden, wo sie Reisfelder mit Fischzucht kombinierten. Ein weiteres Beispiel sind die von den Azteken entwickelten „Chinampas“, künstliche Anbaufläche in flachen Seen.\footcite[Vgl.][S.838]{Turcios2014SustainableFuture} Erste wissenschaftliche Untersuchungen zu dieser Anbaumethode entstanden laut Diver in den achtziger Jahren an der North Carolina State University.\footcite[Vgl.][S.4]{Diver2006Aquaponics-IntegrationAquaculture} 
Ihm zufolge habe dieses System viele Vorteile. Zum einen stellt Diver fest, dass Wasserresourcen geschont werden. Für die Fischzucht war in diesem Fall nur 1 \% der Wassermenge konventioneller Zucht erforderlich. Daher eigne sich das System besonders für klimatisch trockene Gebiete. Ebenfalls würden die Kosten geringer ausfallen. Laut Experimenten an der University of Virgin Islands konnte von den aquaponischen Systemen ein vielfaches mehr an Gemüse geerntet werden als bei feldbasiertem Anbau.\footcite[Vgl.][S.7f]{Diver2006Aquaponics-IntegrationAquaculture} Auch Goddek. et. al. sehen Chancen dieser flexiblen Anbauform für Städte:

\begin{displayquote}
„Aquaponic systems can be set up almost everywhere and have the potential to (sub-)urbanize food production. This could bring important socio-environmental benefits. Aquaponic farming plants could be implemented in old industrial neglected buildings with the advantages of re-establishing a sustainable activity without increasing urbanization pressure on land. Roof gardens would be another possibility, allowing the saving of space in urban areas.“\footcites[S.4214]{Goddek2015ChallengesAquaponics}
\end{displayquote}
Gleichzeitig gibt es eine Reihe an Herausforderungen im Bereich der Aquaponik. Wegen des bestehenden Kreislaufssystems sei der Einsatz von Pestiziden Frei et. al. nach verboten.\footcite[S.43]{FreiMatthiasHartmann2007AquaponikGemuse} So bestehe laut Spring eine hohe Anfälligkeit für Schädlinge, da in häufigen Fällen Monokulturen eingesetzt werden.\footcite[S.27]{Spring2012DerBasel-Stadt} 


\begin{figure}[htbp]
\centering

\includegraphics[width=12cm]{image_folder/Aquaponics_with_catfish.jpg}
\caption{Aquaponiksystem mit Welsen}
\label{fig:aquaponik}
\end{figure}

\subsection{Indoor Farming}

Das Konzept der Landwirtschaft innerhalb von Gebäuden ist eine gängige Methode Nutzpflanzen vor Witterung und Schädlingen zu schützen. Hier werden Pflanzen innerhalb von oft technologisch weit entwickelten Gebäuden unter Anwendung von aufwändigen Bewässerungssystemen und LED-Licht herangezogen. Das Verfahren wird oft mit dem Konzept des Vertical Farmings kombiniert um noch mehr Platz zu sparen. Das Indoor Farming spielt eine immer größere Rolle in der heutigen Zeit, da auf diese Weise Pflanzen an fast allen Orten der Welt angebaut werden können und das unabhängig von Klima und Sonnenstrahlung.

Die Unempfindlichkeit gegenüber den Wetterbedingungen macht Indoorfarming zu einer Anbaumethode mit Zukunftspotential. Städte, die mit Indoor-Farmen ausgestattet sind, könnten weniger anfällig gegenüber Ausnahmesituationen sein. Zudem wird die Nahrungsmittelversorgung auch in Krisenzeiten bei Dürren oder Kälte garantiert. Besonders ihre Unempfänglichkeit gegenüber wechselnden Wetterbedingungen sei ein signifikanter Vorteil. Denn jährlich zerstören Dürre oder Stürme weltweit Ernten und verringern große Erträge. In Anbetracht des Klimawandels ist Indoor Farming daher besonders wichtig für die Ernährungssicherheit von Städten.\footcite[Vgl.][S.27]{Al-Kodmany2018TheCity} Der offensichtliche Nachteil dieser Methode besteht im Energieverbrauch. Eine interessante Fragestellung wäre allerdings, ob der höhere Energiebedarf, solange er aus endlichen Ressourcen gedeckt wird, im Gebäude gegenüber der Sonnenstrahlung im Freien einen höheren Nachteil bildet als die Verbrennung von Fossilen Brennstoffen durch schwere Landmaschinen. 

\subsection{Beispiele}

Im Rahmen dieser Forschungsarbeit wurden Beispielprojekte ausgewählt und bezüglich ihres Beitrags zur nachhaltigen Entwicklung untersucht. Im Folgenden werden die Projekte mit ihren jeweiligen Charakteristiken vorgestellt.

\subsubsection{Growing Underground in London}

Im Londoner Stadtteil Clapham befindet sich die erste Untergrund-Farm der Stadt. Diese liegt 33 m unter London und wurde in einem Lufschutzbunker des zweiten Weltkriegs von den Gründern Richard Ballard, Steven Dring und Michel Roux Jnr eingerichtet. Die Farm baut sogenanntes Mikrogemüse an, welches nach nur wenigen Tagen abgeerntet wird. Die Anbaufläche beläuft sich dabei auf ca. \\10000 m\textsuperscript{2} und die Farm produziert laut Angaben des Unternehmens ca. zwischen \\5.000 und \\20.000 kg Lebensmittel pro Jahr. Aufgezogen werden die Pflanzen mit LED-Licht, für die Bewässerung sorgt ein hydroponisches Bewässerungssystem, welches laut des Unternehmens ca. 70 \% weniger Wasser verbraucht als herkömmliche Anbauformen auf dem Land. Die Fläche des Schutzbunkers blieb bis zur Einrichtung der Farm ungenutzt weswegen keine weitere Fläche erschlossen werden musste. Pestizide und Klima spielen für diese Anbauform keine Rolle da ein eigenes Klima innerhalb der Räume herrscht. Unterstützung erhält das Projekt durch die Universität Cambridge welche die Daten zur Feuchtigkeit, Temperatur und Wachstumsgeschwindigkeit wissenschaftlich auswertet.\footcites[Vgl.]{Lepies2015AusTisch}{Steiniger2015GrowingKam}{2017GrowingLuftschutzbunker}

\subsubsection{Sky Greens in Singapur}

Singapur besitzt im Verhältnis zu seiner Einwohnerzahl (5 Millionen) eine relativ geringe Fläche.
Nur etwa 7\% der Nahrungsmittel, die dort konsumiert werden, wurden auch tatsächlich in Singapur produziert. Nur 250 Acker auf der Insel widmen sich dem Anbau von Nahrungsmitteln. Der weitere Lebensmittelbedarf wird über Import gelöst. Das macht das Land zwangsläufig abhängig und für einen Nahrungsmittelmangel anfällig, falls es zu Lieferenpässen kommt. Aus diesem Gründ hat Singapur viel Mühe in die Forschung des Lebensmittelanbaus mittels \acs{vl} gesteckt.\\ 
\\
Al-Kodmany zufolge ist Sky Greens Singapurs erster kommerzieller urbaner vertikaler Landwirtschaftsbetrieb. Ziel ist eine nachhaltige Produktion von sauberem und frischem Gemüse bei minimalem Land-, Wasser- und Energieressourcenverbrauch" zugewährleisten.\footcite{SkyGreens} Die Farm ist mittlerweile fünf Jahre alt. Sie besteht aus drei Stockwerken und ist damit 9 m hoch. Für den Anbau werden durchsichtige Gewächshäuser verwendet, sodass tropisches Blattgemüse das ganze Jahr über angebaut werden kann. Das Ergebnis sind deutlich höhere Erträgen als dies bei traditionellen Anbau der Fall wäre. Sky Greens produziert jeden zweiten Tag eine Tonne frisches Gemüse. Es liefert eine Vielzahl von tropischem Gemüse, darunter Chinakohl, Spinat, Salat, Xiao Bai Cai, Bayam, Kang Kong, Cai Xin, Gai Lan und Nai Bai. 

Durch die Bereitstellung hochwertiger Produkte zu relativ erschwinglichen Preisen hat sich das Unternehmen entsprechend entwickelt und beabsichtigt in Zukunft seine Produktion zu erweitern. Das Anbausystem besteht aus hohen Aluminiumrfkahmen, die bis zu 9 m hoch sind. Darin verankert sind 38 Ebenen mit Kulturtrögen, die verschiedene Nährböden und Hydrokulturen enthalten.\footcite{SkyGreens} Die Tröge drehen sich mit drei Umdrehungen pro Tag langsam um den Aluminiumrahmen und gewährleisten so, dass die Pflanzen ein gleichmäßiges Sonnenlicht erhalten. Auf diese Weise werde laut Angaben des Herstellers der Bedarf an künstlicher Beleuchtung in vielen Bereichen des Gebäudes reduziert.\footcite{SkyGreens}

\begin{figure}[htbp]
    \centering
    \includegraphics[width=12cm]{image_folder/plantagon.png}
  \caption{Plantagon - Erster Prototyp für tropische Städte}
  \label{fig:plantagon}
\end{figure} 

\subsubsection{Plantagon}

Gegründet im Jahr 2008 in Schweden ist Plantagon, ein internationales Unternehmen, das durch sein innovatives Vertikal-Farming-Gebäudekonzept viele Preise gewonnen hat.\footcite[Vgl.]{PlantagonAwardsPlantagon} Besonders innovativ sei ihr Konzept zum automatisierten Lebensmittelanbau (siehe Abbildung \ref{fig:plantagon}), das wie folgt funktioniere:\footcite[Vgl.][S.21f]{Al-Kodmany2018TheCity}  In der Mitte eines Gebäudes befindet sich eine Helixstruktur. Am unteren Ende der Helix werden die Samen in Töpfe eingesäht. Daraufhin werden mit Hilfe eines Aufzuges diese Töpfe ans oberen Ende der Spirale befördert. Hier wird jede Pflanze entlang der Helix einzeln durch robotische Förderbänder abhängig vom Sonnenlicht, dem Alter und ihrer Größe hoch und runter bewegt. Anschließen werden sie dann zur Ernte nach unten befördert. Dieses Konzept könne in drei verschiedenen Gebäudeformen umgesetzt werden: Entweder integriert an einer Gebäudefassade, als multifunktionaler Gebäudekomplex mit Büros, Hotels und Mietswohnungen oder in einem Gebäude, das sich ausschließlich der landwirtschaftlichen Produktion widme. Für die letzte Variante wurden zwei Prototypen erstellt. Der erste Prototyp (siehe Abbildung \ref{fig:plantagon}) hat die Form einer Halbkugel, der zweite Prototyp „Plantscraper“ diene als aktuelles Referenzmodell für ein \acs{vl}-Konzept (siehe Abbildung \ref{fig:plantagon_2}). Dieses Gebäude umfasst zwölf Etagen, einen „Bauernmarkt“ im Erdgeschoss und Büroräume für weitere \acs{ul}-Forschung. Ausgerichtet auf asiatische Megastädte, soll die Produktion schätzungsweise 300 bis 500 Tonnen Blattgemüse wie beispielsweise Pak-Choi produzieren. Trotz mehreren Preisen und angemeldetem Patent existiert derzeit noch kein umgesetztes Hochhaus nach diesem Konzept. Die Angaben zum Ertrag und Erfolg von Plantagon bleiben also kritisch zu betrachten.

\begin{figure}[htbp]
    \centering
    \includegraphics[width=9cm]{image_folder/plantscraper.png}
  \caption{Referenzmodell „The Plantscraper“ von Plantagon}
  \label{fig:plantagon_2}
\end{figure} 

\subsection{Nachhaltigkeitsbewertung}
%Lauras Nachhaltigkeitsbewertung zu \acs{vl}
Zu den genannten Beispielen vertikaler Landwirtschaft wurden in der Literatur hauptsächlich positive Aspekte dargestellt. Dennoch ist eine kritische Auseinandersetzung Teil dieser Forschungsarbeit. Im Folgenden wird die vertikale Landwirtschaft hinsichtlich ihres Beitrags zur ökologischen Nachhaltigkeit abschließend bewertet.

In Kapitel \ref{UnsereKriterien} wurden eigene Kriterien zur Bewertung der ökologischen Nachhaltigkeit aufgestellt. Als wichtig erachtet wurden dabei unter anderem der Schutz natürlicher Ressourcen wie Boden, Wasser oder Luftqualität, die Nutzung regenerativer Energie, sowie die Schadstoffreduktion durch Emissionenen, Abfall oder Pestizide. Diese Kriterien dienen im Folgenden als Orientierung für die Bewertung.
\\
\\
\textbf{Ökonomische Wirtschaftlichkeit}\\
Im Vergleich zu konventionellem Anbau sind die Produktionskosten bei \acs{vl} unabhängig vom Ölpreis (Siehe \ref{Vergangenheit der Urbanen Landwirtschaft} am Beispiel Kuba), auf diese Weise sind die Preise der Lebensmittel unabhängig von eventell steigenden Transportkosten. Gleichzeitig können Indoor-Farming-Systeme aufgrund ihrer Abgeschlossenheit ganzjährig Nahrungsmittel produzieren und sind durch ihre Abgeschlossenheit unabhängig vom Jahreswechsel oder extremen Klimabedingungen. Durch die Anwendung von Anbaumethoden wie Hydroponik, Aquaponik oder Aeroponik können höhere Erträge erreicht werden. Besonders in Aquaponik sehen Bliladriu et. al. trotz größerer Einstiegshürde ein rentables Geschäft des Systems:
\begin{displayquote}
„This type of agriculture might mean a stepped-up investment, but it is one that creates another revenue stream (from fish) linked with more profitable plant production. That means greater financial resiliency for a business – and maximizing dollar returns to shareholders can be a very powerful force far rapid change.“\footcite[Vgl.][S.6]{Blidariu2011NcreasingAquaponics-Review}
\end{displayquote}
Besonders der Anbau in die Höhe und die daraus resultierende Ertragssteigerung pro Quadratmeter ist ein wesentlicher Vorteil von \acs{vl}.\footcite{Despommier2010TheCentury.} \\
\\
\textbf{Rücksicht Dritter}\\
Bei vertikaler Landwirtschaft ist die Gefahr geringer, dass Keime oder Schädlinge im großen Umfang über die Erde weitergegeben werden. Bei Aquaponik könne eine bequeme Arbeitshöhe an dem System eingestellt werden, was besonders vorteilhaft für ältere Menschen oder Menschen mit Behinderung sei\footcite[Vgl.][S.6]{Blidariu2011NcreasingAquaponics-Review}. Ebenfalls kommt die geringe Instandhaltung und Pflege \textit{Dritten} zugute.
\\

%\begin{displayquote}
%„Vertical farming meets the needs of an increasing urbanization. Buildings used for farming can be placed anywhere while outdoor fields are static in location. By strategically placing vertical farms inside or in the near vicinity of urban centers and cities, we are able to meet the need for localization of food production. Foods can be harvested and sold in the same building immediately after harvest, eliminating the need for transportation and storage.”\footcite[S.7]{PeterMollerVoss2013VerticalRise}
%\end{displayquote}

\textbf{Langfristigkeit}\\
Gemäß Schulz erscheint „Besonders im Hinblick auf prognostizierte Wetterextreme im Zuge des Klimawandels und der besonderen Betroffenheit der Landwirtschaft [...] das isolierte Wirtschaften vorteilhaft.“\footcite{Schulz2013UrbaneLandmanagements}\\
\\
\textbf{Schutz der Umwelt}\\
Vertikale Landwirtschaftsbetriebe benötigen keinen bestehende Landfäche, auf diese Weise wird keine Naturfläche für den Zweck des Lebensmittelanbaus degradiert. Allerdings entstehen durch den Bau der Landwirtschaftbetriebe weitere Umweltbelastungen. \\
\\
\textbf{Förderung der Biodiversität}\\
Laut Wirz et al. sei die Umgestaltung natürlicher Ökosysteme in Agrarflächen die Hauptursache für den weltweiten Rückgang der Artenvielfalt.\footcite[S.18]{Wirz2017Okologisierte2050} Ein positiver Beitrag zur Biodiversität wird von vertikalen Landwirtschaftsbetrieben aktiv nicht geleiset. Lediglich wird durch die Nichtnutzung von ländlicher Ackerfläche keine Degradation bestehender Böden verursacht.
Wird vertikale Agrikultur beispelsweise an Hochhausfassaden, also außerhalb betrieben, wirkt sich dies positiv auf die Luftfeuchtigkeit, die Temperatur und den CO\textsuperscript{2} aus, was eine unterstützende Wirkung auf die Biodiversität der Stadt besitzt.\\
\\

\textbf{Gesellschaftliche Akzeptanz}\\
Die Akzeptanz für vertikale Landwirtschaft sei laut Despommier bei vielen Stadtbewohnern nicht vorhanden, da diese nicht mehr an die Landwirtschaft erinnere, die sie zuvor kannten.
%So ist allein der Begriff „Vertical Farm” bei großen Teilen der Bevölkerung negativ belegt, da dieser Assoziationen von „künstlichem und unnatürlichem Anbau” wecke.
Ebenfalls werde im Schnitt 30\% der Lebensmittel herkömmlicher Landwirtschaft während Transport und Lagerung durch Druckstellen oder Fäule unverkäuflich.\footcite{Despommier2009TheFarms}
Nahrungsmittel die  mittels vertikaler Landwirtschaft angebaut wurden gelangen dank der kurzen Distanz zwischen Produktion und Konsumenten meist frischer zum Konsumenten, wodurch der Nährstoffgehalt häufig größer ausfalle.\footcite{Despommier2009TheFarms}\\
\\
\textbf{Schadstoffreduktion}\\
Während Lebensmittel aus traditionellem Anbau im Durchschnitt ungefähr 2400 km zurücklegen, bis sie zum Konsumenten gelangen, ermöglicht \acs{vl} verbrauchernahe Produktion. Der Wegfall von langen Transportwegen und unnötiger Lagerung, kann somit den Ausstoß von CO\textsuperscript{2} verringern. 
%In einem weiteren Schritt könne dann daran gedacht werden nach Bedarf zu ernten, was unnötige Lagerung sowie Lebensmittelabfall noch weiter reduzieren könne.
Laut Voss sei ein weiterer Vorteil die Ortsunabhängigkeit. \footcite[S.7]{PeterMollerVoss2013VerticalRise} Während landwirtschaftliche Flächen ortsgebunden und abhängig von der Bodenqualität sind, können \acs{vl} überall gebaut werden, wo Quadratmeter zur Verfügung stehen. So eigne sich \acs{vl} seiner Meinung nach besonders, um die Probleme der steigenden Verstädterung zu mildern. Sofern Aquaponik als Anbaumethode verwendet wird, werden in diesem Fall Pestizide oder künstlicher Dünger gemieden, was die Verwendung von Schadstoffen reduziert.\footcite[Vgl.][S.10]{Al-Kodmany2018TheCity}\\
\\
\textbf{Nutzung regenerativer Energien}\\
Im Vergleich zu traditionellen und landbasierten Anbaumethoden, die für Traktoren, Pflüge etc. auf fossile Energien angewiesen sind\footcite[]{2012TheEssay}, kann \acs{vl} auf regenerative Energie zugreifen. So entstehe laut Sharanbir et al. durch den Transport \footcite[Vgl.][]{GrewalCanFood}, die Lagerung und Produktion ein hoher fossiler Verbrauch. Beispielsweise werde in Nordamerika 20\% der fossilen Energie allein von der Landwirtschaft verbraucht.\footcite[Vgl.][S.27]{Al-Kodmany2018TheCity}\\
\\
\textbf{Sparsamer Ressourcenumgang von Anbau bis Konsum}\\
Einige \acs{vl}-Systeme können Grau- und Schwarzwasser\footnote{Als Grau- und Schwarzwasser wird häusliches Abwasser bezeichnet.} wiederverwenden.
Möller Voss bezieht sich in seiner Examensarbeit auf die Aussage Despommiers. \footcite[S.9]{PeterMollerVoss2013VerticalRise} Demnach bestehe durch eine Nutzung von Regenwasser oder städtischen Grauwassers für vertikale Farmen die Möglichkeit autark im Bereich der Wassernutzung zu werden.\footcite[Vgl.]{Despommier2010TheCentury.} Diese Aussage sollte allerdings mit Abstand betrachtet werden, da der nötige Energieaufwand für die Technologie der Wasseraufbereitung schlussendlich entscheidet, ob unter dem Strich Ressourcen eingespart werden. Im Hinblick auf die genannten Anbaumethoden wird gegenüber dem herkömmlichen Feldanbau wesentlich viel Wasser gespart. Die Aeroponikmethode ist beispielsweise besonders stark auf die Wasserersparnis optimiert, da sie Wasserdunst nutzt, der durch die Abgeschlossenheit des Systems wiederverwendet werden kann.\footcite[Vgl.][S.8f]{Al-Kodmany2018TheCity} Aquaponik ist im speziellen ressourcensparend, da das Kreislaufsystem zum einen Nährstoffe für die Pflanzen generiert und zum anderen das Wasser zur Wiederverwendung reinigt.\footcite[Vgl.][S.10f]{Al-Kodmany2018TheCity}\\
\\
\textbf{Kreislaufsystem}\\
Biomasse, die bereits zur Wiederaufbereitung von Wasser verwendet wurde, sowie weitere Pflanzenabfälle können als Kraftstoff zur Eigenbetreibung und Betreibung weiterer Landwirtschaftsbetriebe verwendet werden.\footcite[Vgl.][S.80ff]{Despommier2009TheFarms} Weiterhin sind Aquaponiksysteme wie zuvor genannt selbst effziente Kreislaufsysteme sind. 






%So erfüllt \acs{vl} die Kriterien der Nachhaltigkeit unter folgenden Bedingungen:
%Für den Anbau und die Produktion muss regenerative Energie genutzt werden. Erst so kann die Ressourceneffizienz der Lebensmittel im Vergleich zu Nahrung aus konventionellem Anbau reduziert werden. 
%Im Idealfall sollte also ein vertikaler Landwirtschaftsbetrieb kostengünstig gebaut werden, dauerhafte und sichere Arbeitsbedingungen schaffen und unabhängig von externen finanziellen Förderungen oder anderer Unterstützung, damit eine langfristige und unabhängige Nahrungsmittelproduktion bürgernah und zentral bestehen kann.\footcite{Despommier2010TheCentury.}\\

\begin{figure}[htbp]
\centering
\includegraphics[width=14cm]{image_folder/automatisation_kondo.png}
\caption{Eine Ernte-Roboter bei der Arbeit}
\label{fig:Automatisierung}
\end{figure}
\\

\subsection{Kritische Auseinandersetzung mit \acs{ul} }


Sofern die Idee von \acs{vl} darin besteht Nahrungsmittelproduktion zu lokalisieren, muss der vertikale Landwirtschaftsbetrieb dementsprechend zentral in der Stadt platziert sein. Eine der großen Herausforderungen, die sich hierbei für die Planer von \acs{vl} - Projekten ergibt, ist die Verfügbarkeit von erschwinglichem Bauplatz in der Stadt. So scheitern laut Möller viele der Projekte vor dem Start aufgrund von fehlenden finanziellen Mitteln.\footcite[S.8]{PeterMollerVoss2013VerticalRise} 

Abschließend werden die Vor- und Nachteile von \acs{ul} wie folgt zusammengefasst.

\textbf{Vorteile von \acs{ul} für Städte} \\
Sofern Landwirtschaft im urbanen Raum und außerhalb (d.h nicht in Fabriken oder abgeschlossenen Räumen) betrieben wird, wirkt sich dies über die Beeinflussung der Temperatur, Luftfeuchtigkeit und CO\textsuperscript{2}- Aufnahme positiv auf das Mikroklima und die Biodiversität der Stadt aus.
Gleichzeitig kann durch einen vermehrten Pflanzenbewuchs in der Stadt (vor allem bei Bewuchs um Gebäude) eine Milderung des sogenannten Urban-Heat-Effects entstehen. Das bedeutet, die Gebäude heizen sich im Sommer also weniger auf, was sich wiederum positiv auf die Luftqualität auswirkt und Energiekosten für eventuelle Klimatisierung eingespart werden können.\footcites{Schulz2013UrbaneLandmanagements}[Vgl.][S.16]{Spring2012DerBasel-Stadt}

\begin{figure}[htbp]
\centering
\includegraphics[width=14cm]{image_folder/urbanheat.png}
\caption{Der Urban Heat Effekt}
\label{fig:urbanheateffekt}
\end{figure}
\\
Urbaner Anbau von Lebensmittel und folglich kurze Entfernungen zum Verbraucher, bieten neue Möglichkeiten zur Reduzierung von Transportwegen, Kühlung und Verpackung von Lebensmitteln. Der Bedarf an fossilen Rohstoffen kann auf diese Weise sinken, was sich positiv auf die Bewertung des ökologischen Fußabdrucks der Lebensmitteln auswirkt. Schulz fasst viele wichtige Punkte zusammen: „Kurze, innovative Nahrungsverteilungswege zwischen Erzeugern und Verbrauchern, Transporteinsparungen, frische Produkte, Integration der „Landwirte“ in Marketingprozesse sowie bessere Absprache zwischen Angebot und Nachfrage sind potenzielle Vorzüge gegenüber einer globalisierten Landwirtschaft.“\footcite[S.10]{Schulz2013UrbaneLandmanagements} Die starke lokale Nähe zwischen Produktion und Verbrauch lässt ebenso Platz für zukünftige Pläne eines Anbaus je Bedarf. Auf diese Weise könnte unnötiger Lebensmittelabfall verhindert werden. Die Integration in die Stadt ermöglicht auch eine Nutzung der städtischen Ressourcen wie Abwärme oder Abwässer, Es entsteht ein Kreislauf-Effekt der ebenso zur Reduzierung des Bedarfs externen Rohstoffe beiträgt.
Sofern für die Zwecke der urbanen Landwirtschaft innerstädtische Flächen entsiegelt werden, kann eine Verbesserung im urbanen Wassermanagement erreicht werden. Dies kann eine positive Auswirkung auf die Grundwasserneubildung haben und das Kanalsystem kann somit entlastet werden.  
  

\begin{figure}[htbp]
\centering
\includegraphics[width=14cm]{image_folder/moeglicheVorteileUL.png}
\caption{Mögliche Vorteile von \acs{ul} für Städte}
\label{fig:MoeglicheVorteilevonUL}
\end{figure}

\\


     
\textbf{Nachteile von \acs{ul}}

Da große Arbeitsgeräte wegen der häufig sehr engen Umgebung in vertikalen Landwirtschaftsbetrieben nicht eingesetzt werden können (z.B. aufgrund von geringer Wegbreiten), kann häufig ein hoher manueller Arbeitsaufwand entstehen.
Obwohl VL einen höheren Ertrag pro Fläche erziele, wäre der Ertrag im Vergleich zum großflächigen Anbau auf kommerziellen Großbetrieben nur gering.\footcite[Vgl.][S.8f]{BRINK2002LandwirtschaftsprogrammHannover} So sei der Gesamtertrag zu gering um den Bau eines Vertikalen Landwirtschaftsbetriebs zu refinanzieren. Laut Brink sei vielfach erst eine Schlaggröße von vier bis fünf Hektar für den Landwirt wirtschaftlich interessant.“ \footcites[S.27]{Schulz2013UrbaneLandmanagements}[zitiert nach]{STADTENTWICKLUNG2010OffenerLandwirtschaft}

Desweiteren werden in einem Großteil der Literatur, die sich mit vertikaler Landwirtschaft beschäftigt, die Vorteile dieser Anbauform überladen. 
Despommier behauptet so, dass ein vertikaler Landwirtschaftsbetrieb mit 30 Stockwerken, heutiger Technologie und der Größe von einem Block von ungefähr 278709,12 m\textsuperscript{2} (3 millionen square feet) bis zu 10.000 Menschen versorgen könnte.\footcite{Despommier2010TheCentury.}

\begin{displayquote}
„Working within the framework of these calculations, one vertical farm with an architectural footprint of one square city block and rising up to 30 stories (approximately 3 million square feet) could provide enough nutrition (2,000 calories/day/person) to comfortably accommodate the needs of 10,000 people employing technologies currently available.” 
\end{displayquote}

Wird diese Rechnung einmal
%(unter Berücksichtigung, dass 3 millionen square feet in etwa 278709,12 m\textsuperscript{2} betragen)
nachvollzogen, bedeutet dies Folgendes:

Die gesamte Anbaufläche eines dementsprechenden Betriebs entspräche\\278709,12 m\textsuperscript{2}
Da diese Fläche auf 30 Stockwerke aufgeteilt ist, wäre die Grundfläche des Betriebs 9290.304 m\textsuperscript{2} groß (also
278709,12 m\textsuperscript{2}/ 30 Stockwerke = 9290.304 m\textsuperscript{2}). Diese Grundflundfläche entspricht in etwa 1,3 Fußballfelder.

Am Beispiel Tokio, einer Stadt mit mehr als 9.00.000 Einwohnern auf einer Fläche von ungefähr 622000 m\textsuperscript{2} (ungefähr 87 Fußballfelder) würde dies bedeuten, dass sofern ganz Tokio mit vertikalen Landwirtschaftsbetrieben bestückt wäre, lediglich 10\% der gesamten Stadtbevölkerung gesättigt würde.
Dies spricht eindeutig dafür, dass Despommiers Schlussfolgerung einer erneuten kritischen Beurteilung bedarf.
 
Da die vertikale Landwirtschaft auf künstliches Licht angewiesen ist, um Pflanzen anzubauen, sei der Stromverbrauch in den jeweiligen Betrieben sehr hoch, welcher zu einer erhöhten Umweltverschmutzung und Emissionen von Treibhausgasen führe.
Darüber hinaus seien die Kosten für den Kauf der in der vertikalen Landwirtschaft verwendeten LED-Leuchten für viele kleine und mittlere Unternehmen kaum bezahlbar.
Ebenso wird von Kritikern argumentiert, dass die Pflanzen, die vertikal angebaut werden können, das Ausmaß der Umweltprobleme nicht lösen können und somit die vertikale Landwirtschaft keine geeignete Lösung für die bestehenden landwirtschaftlichen Probleme darstelle.\footcites[Vgl.][S.1]{Coyle2017WillReach}

Die Swot-Analyse von Schulz bewertet VL hinsichtlich der ökonomischen Wirtschaftlichkeit wie folgt:
 
\begin{figure}[htbp]
\centering
\includegraphics[width=14cm]{image_folder/swot_kSchulz.png}
\caption{SwotAnalyse zur Bewertung der ökonomischen Tragfähigkeit von \acs{ul}}
\label{fig:SwotUL}\footcite[S.32]{Schulz2013UrbaneLandmanagements}
\end{figure} 
 
\section{Prognose zur urbanen Landwirtschaft}


\subsection{Zukünftige globale Landnutzung}

Um herauszufinden wie wichtig die Entwicklung von urbaner Landwirtschaft tatsächlich ist, ist es zunächst bedeutend Landwirtschaftsformen auf deren Grundlage zu untersuchen. Die Entwicklung der globalen Landnutzung ist insofern relevant, dass die Landressourcen für eine Versorgung durch Lebensmittel entscheidend sind. Die Versorgung und deren Sicherstellung hängen wiederum stark von der Weltbevölkerung ab. Das Bundesministerium für Umwelt, Naturschutz und nukleare Sicherheit, der Industrieverband Agrar e. V. (IVA) und die \acs{fao} geben an, dass die Weltbevölkerung bis zum Jahr 2050 auf über neun Milliarden Menschen ansteigt.\footcites[Vgl.]{BMU2016GlobaleUmweltfolgen}[sowie]{Agrar0NahrungsmittelLandwirtschaft}{FAO2009How2050} Laut dem Bundesministerium für Umwelt bereite die Lebensmittelproduktion aber bereits jetzt Umweltprobleme. Zudem würden bereits ca. 800 Millionen Menschen an Hunger leiden.\footcite[Vgl.]{BMU2016GlobaleUmweltfolgen} \\
\\
Der IVA prognostizierte eine pro Kopf Anbaufläche von 1700 m\textsuperscript{2} im Jahre 2050. 2010 waren es laut dem IVA noch ca. 2100 m\textsuperscript{2} pro Person.\footcite[Vgl.]{Agrar0NahrungsmittelLandwirtschaft} Die \acs{fao} ging 2009 zudem davon aus, dass fast der gesamte Bevölkerungszuwachs in den Entwicklungsländern stattfinden wird. Die Urbanisierung würde sich fortsetzen und beschleunigen und etwa 70 \% der Weltbevölkerung würden städtisch sein. Um diese größere, städtischere und reichere Bevölkerung zu ernähren, müsse die Nahrungsmittelproduktion um 70 \% steigen. Die jährliche Getreideerzeugung müsse auf rund 3 Milliarden Tonnen (2,1 Milliarden Tonnen, Stand: 2017 \footcite[Vgl.]{IGCInternationalGrainsCouncil2018IGCIndex}) steigen, und die jährliche Fleischproduktion müsse um über 200 Millionen Tonnen auf 470 Millionen Tonnen steigen.\footcite[Vgl.][S. 2]{FAO2009How2050}


%Die Fleischproduktion betrug laut der \acs{fao} 2017 ca. 330.4 Millionen Tonnen. Gleichzeitig prognostiziert die \acs{fao} bis Ende 2018 einen Anstieg auf 336.2 Millionen Tonnen.\footcite[Vgl.][S. 7]{} Das statistische Bundesamt veröffentlichte 2010 einen Bericht zur Flächenbelegung von Ernährungsgütern in dem die Flächennutzung für Nutzvieh weltweit auf ca. 13.123.000 Hektar Land geschätzt wird. Das entspricht einer Fläche von ca. 131.230.000.000 m\textsuperscript{2}. Der Futterverbrauch der Tiere lag dabei bei ca. 130.234.000 Tonnen.\footcite[Vgl.][S. 9]{Sttistisches Bundesamt} Die Menge der Fleischproduktion betrug in diesem Jahr ca. 293 Millionen Tonnen.\footcite[Vgl.]{Institute for agricultere & trade Policy} %
Steigt der Konsum auf über 470 Millionen Tonnen, benötigen wir bei derzeitigen Aufzucht- und Anbaumethoden rein rechnerisch etwa 60 \% mehr Fläche zur Aufzucht von Tieren. In einem Zeitraum von 40 Jahren bedeutet dies einen enormen Anstieg des Flächenbedarfs.

Beim Anbau von pflanzlichen Nahrungsmitteln kann ein ähnlicher Trend verzeichnet werden. Das Bundesministerium für Ernährung und Landwirtschaft gibt den pro Kopf Konsum pflanzlicher Nahrungsmittel im Jahre 2010 auf geschätzte 629,4 Kilogramm an. Dabei beruft es sich auf Quellen der \acs{fao}, Eurostat und BLE.\footcite[Vgl.][S. 1]{BundesministeriumfurErnahrungundLandwirtschaft2005DurchschnittlicherKopf} Die UN gibt die Höhe er Weltbevölkerung 2010 auf ca. 6,92 Milliarden Menschen an.\footcite[Vgl.][S. 3]{UnitedNationsDepartmentofEconomicandSocialAffairs2017World2017}Errechnet man den Verbrauch der Menschen weltweit für das Jahr 2010 ergibt sich die Summe von 4,361 Billionen kg. Geht man von den Prognosen zur Weltbevölkerung im Jahr 2050 aus, so benötigen ca. 9,725 Milliarden Menschen schätzungsweise 6,121 Billionen kg. Bewahrheiten sich die Prognosen werden 2050 40 \% mehr pflanzliche Nahrungsmittel benötigt als 2010. Würden sich die Anbaumethoden für pflanzliche Nahrungsmittel zudem nicht weiterentwickeln, würde demnach auch mehr Landwirtschaftsfläche benötigt.\\
\\
Dass Projekt „Weltacker“ der Zukunftsstiftung Landwirtschaft versucht aus diesem Grund in Brandenburg herauszufinden wie viel Nahrungsmittel und Fläche ein Mensch benötigt um sich selbst zu versorgen. Auf einer Fläche von von 2000 m\textsuperscript{2} wird angebaut was ein Mensch theoretisch benötigt, darunter Nahrungsmittel aber auch Rohstoffe für Kleidung und erneuerbare Energien. Svenja Nette welche Qualitative Forschung im Bereich Landwirtschaft und Umwelt studierte, spiegelt in Ihrem Bericht „Wie viel Erde braucht der Mensch?“ ihre Erfahrungen mit dem Projekt wieder. Der „Weltacker“ steht dabei für jenen Anteil der Weltanbaufläche der einem einzelnen Menschen rein theoretisch zusteht. Ihren Angaben zufolge könne ein Mensch mit ca. 650 Quardratmetern Grundfläche durch vegetarische Ernährung auskommen. \footcite[Vgl.]{Nette2014WievielMensch}
%Laut der deutschen Gesellschaft für Ernährung benötigt ein erwachsender Mann im Durchschnitt 2300 kcal, eine erwachsene Frau 1800 kcal durchschnittlich, beide im Alter von 25-50 Jahren bei geringer körperlicher Aktivität. Das Essverhalten eines Menschen könne aber dennoch stark variieren.\footcite[Vgl.]{Wie viel Energie braucht der Mensch?} %
Laut Nette benötige ein Europäer im Durchschnitt aber 2700 m\textsuperscript{2} an Anbaufläche um seinem Ess- und Konsumverhalten gerecht zu werden. Das wären bereits jetzt 700 Quardratmeter mehr als ihm rein rechnerisch zustünde.\footcite[Vgl.]{Nette2014WievielMensch} Die Konsiquenz aus diesem Konsumverhalten würde bedeuten, dass den Menschen andererorts die nötige Anbaufläche fehlt. Beachtet man die wachsende Weltbevölkerung, verringert sich die Anbaufläche pro Person zusätzlich in den kommenden Jahren und könnte in der Zukunft kritische Ausmaße erreiche.\\
\\
Es bleibt festzustellen, dass Anbaumethoden von Nahrungsmitteln und Rohstoffen, welche eine geringe Anbaufläche benötigen an Relevanz gewinnen um auch zukünftig die Nahrungsversorgung sicherzustellen. Ferner bleibt festzuhalten, dass eine Nutzung von Brachflächen und der Einzug der Landwirtschaft in urbane Gebiete neue Anbauflächen generieren können. Die Erschließung neuer ruraler Anbaufläche würde bedeuten, weiterhin natürliche Lebensräume anderer Arten in Ackerflächen zu wandeln. Geht diese Entwicklung weiter so wird die Artenvielfalt zwangsläufig gefährdet.

\subsection{Zusammenfassung}

\section{Fazit}

Verschiedene Prognosen lassen 
\section{Danksagung}

\newpage
\listoffigures

\begingroup
\singlespacing
\setlength\bibitemsep{10pt} % oder was auch immer
\printbibliography
\endgroup

\end{document}
